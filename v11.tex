\wdh
Eine Abbildung $f:x→y$\\
\begin{itemize}
\item{ist \underline{injektiv} wenn gilt:\\
für alle $a,b\in X$ mit $f(a)=f(b)$ ist $a=b$}
\item{ist \underline{surjektiv} wenn für jedes $y\in Y$ ein $a\in X$ existiert mit $f(a)=y$}
\end{itemize}}
Sei $D\subseteq\R$ Teilmenge. Eine Funktion auf D ist eine Abbildung $f:D→\R$
%
\uS{Monotone Funktionen}
\bem
Eine Funktion $(a_n)_{n\geq 0}$ reeler Zahlen ist eine Abbildung $a:\N_0→\Re$ d.h. eine Funktion auf $\No_0$
%
\def
Sei $D\subseteq\R$. Eine Funktion  $f:D→\R$ heißt:
\begin{enumerate}
\item{\underline{monoton wachsend} wenn gilt:\\
Für alle $a,b\in D$ mit $a<b$ ist immer $f(a)\leq f(b)$}
\item{\underline{streng monoton wachsend}: $a<b\Rarr f(a)<f(b)$}
\item{\underline{monoton fallend}: $a<b\Rarr f(a)\geq f(b)$}
\item{\underline{streng monoton fallend}: $a<b\Rarr f(a)> f(b)$}
\end{enumerate}
%
\bem
Jede streng monotone Funktion f ist injektiv
%
\bew
Zeige: $a\neq b\Rarr f(a)\neq f(b)$\\
Wenn $a\neq b$ dann $a< b$ oder $b<a$\\
Wenn f streng monoton wachsend: Folgt $f(a)< f(b)$ oder $f(b)< f(a)$ also $f(a)\neq f(b)$\\
Wenn f streng monoton fällt: es folgt $f(a)> f(b)$ oder $f(b)> f(a)$ also $f(a)\neq f(b)$\qed
%
%\sS{Beispiel}
%\begin{enumerate}
%\item{$f:\R_{\geq 0}→\R,\ x\mapsto x^k =:f(x)$ mit $k\geq 1$\\%umgekehrtes define
%f ist streng monoton wachsend/steigend%bild tafel 2.2
%\end{enumerate}
%
% Cristopher: Tafel 2.2.2 bis 5.2.1
%
\chapter{Stetigkeit}
\underline{Idee:} Eine Funktion ohne sprünge heißt \underline{stetig}\\
\def
Sei $D\subseteq \R,\ f:D→\R$ eine Funktion\\
\begin{enumerate}
\item{f heißt stetig in $x\in D$ wenn gilt:\\
Für jedes $\e>0$ gibt es ein $\delta>0$ so dass für jedes $y\in D$ mit $|x-y|<\delta$ gilt $|f(x)-f(y)<\e$ %graph 6.2}
\item{f heißt stetig wenn f in jedem $x\in D$ stetig ist}
\end{enumerate}
\sS{Beispiel}
\begin{enumerate}
\item{Die Funktion $id:\R→\R,\ x\mapsto x$ ist stetig}
\item{Die Funktion $f:\R→\R,\ f(x)=x^2$ ist stetig. %graph klein
\bew
Sei $x,y\in\R\quad y=x+k$.\\
$$f(y)-f(x)=(x+h)^2-x^2=x^2+2xh+h^2-x^2=2xh+h^2$$\\
Wähle jedenfalls $\delta\leq 1$. Wenn $|h|=|x-y|>\delta$ dann $|h|<1$\\
$$|f(y)-f(x)|=|2xh+h^2|\leq|2x|·|h|+|h|^2<|2x|·|h|+|h|=(|2x|+1)·|h|$$\\
Gegeben sei $\e>0$\\
Wähle $\delta=min\left\{1,\dfrac{\e}{|2x|+1}\right\}$\\
Wenn $|x-y|<\delta$ dann $$|f(x)-f(y)|<(2|x|+1)·|h|<(2|x|+1)·\dfrac{\e}{2|x|+1}=\e$$\\
Also $f$ stetig in $x$}
\item{$g≔\R→\R,\ g(x)≔\{x\}$\\ %graph
g ist stetig an $x$ \equ $x\notin\Z$
\Bew{g nicht stetig an $x\in\Z$:}
Zeige: es gibt ein $\e>0$ so dass kein $\delta>0$ existiert mit: $|x-y|>\delta\Rarr|g(x)-g(y)|<\e$\\
z.B. $\e=1$ Sei $\delta>0.\ y=x-\dfrac{\delta}{2}\quad |x-y|=\dfrac{\delta}{2}<\delta$\\
aber $g(y)=\{x-\dfrac{\delta}{2}\}=x-1$ (weil $x\in\Z$)\\
|g(x)-g(y)|=|x-(x-1)|=1 \not<\e\qed}
\end{enumerate}
%
% Christopher
%
\sS{Satz (Folgenstetigkeit)}
Sei $D\subseteq\R,\ x\in D,\ f:D→\R$ Funktion $f$ ist genau dann stetig in $x$ wenn gilt:\\
\begin{itemize}
\item{Für jede Folge $(x_n)_{n\geq 0}$ mit $x_n\in D,\ x_n→x$ für $n→∞$ gilt auch $f(x_n)→f(x)$ für $n→∞$}
\end{itemize}
%
\satz
Sei $D\subseteq\R,\ f,g:D→\R$ in $x\in D$\\
Dann gilt:\\
\begin{itemize}
\item{$f+g:D→\R$ stetig in $x$}
\item{$f·g:D→\R$ stetig in $x$}
\item{Wenn $g(x)\neq 0$ für alle $x'\in D$}
\end{itemize}
Dann ist $\dfrac{1}{f}:D→\R$ stetig in x.
\bew
mit Folgenstetigkeit
%letzte seite
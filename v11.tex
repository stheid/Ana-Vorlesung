% Kopfzeile beim Kapitelanfang:
\fancypagestyle{plain}{
%Kopfzeile links bzw. innen
\fancyhead[L]{\calligra\Large Vorlesung Nr. 11}
%Kopfzeile rechts bzw. außen
\fancyhead[R]{\calligra\Large 15.11.2012}
}
%Kopfzeile links bzw. innen
\fancyhead[L]{\calligra\Large Vorlesung Nr. 11}
%Kopfzeile rechts bzw. außen
\fancyhead[R]{\calligra\Large 15.11.2012}
% **************************************************
%
\wdh
Eine Abbildung $f:x→y$\\
\begin{itemize}
\item{ist \underline{injektiv} wenn gilt:\\
für alle $a,b\in X$ mit $f(a)=f(b)$ ist $a=b$}
\item{ist \underline{surjektiv} wenn für jedes $y\in Y$ ein $a\in X$ existiert mit $f(a)=y$}
\end{itemize}
Sei $D\subseteq\R$ Teilmenge. Eine Funktion auf $D$ ist eine Abbildung $f:D→\R$
%
\uS{Monotone Funktionen}
\bem
Eine Funktion $(a_n)_{n\geq 0}$ reeller Zahlen ist eine Abbildung $a:\N_0→\R$ d.h. eine Funktion auf $\N_0$
%
\sS{Definition}
Sei $D\subseteq\R$. Eine Funktion  $f:D→\R$ heißt:
\begin{enumerate}
\item{\underline{monoton wachsend} wenn gilt:\\
Für alle $a,b\in D$ mit $a<b$ ist immer $f(a)\leq f(b)$}
\item{\underline{streng monoton wachsend}: $a<b\Rarr f(a)<f(b)$}
\item{\underline{monoton fallend}: $a<b\Rarr f(a)\geq f(b)$}
\item{\underline{streng monoton fallend}: $a<b\Rarr f(a)> f(b)$}
\end{enumerate}
%
\bem
Jede streng monotone Funktion $f$ ist injektiv
%
\bew
Zeige: $a\neq b\Rarr f(a)\neq f(b)$\\
Wenn $a\neq b$ dann $a< b$ oder $b<a$\\
Wenn f streng monoton wachsend: Folgt $f(a)< f(b)$ oder $f(b)< f(a)$ also $f(a)\neq f(b)$\\
Wenn f streng monoton fällt: es folgt $f(a)> f(b)$ oder $f(b)> f(a)$ also $f(a)\neq f(b)$\qed
%
\sS{Beispiel}
\begin{enumerate}
\item{$f:\R_{\geq 0}→\R,\ x\mapsto x^k =:f(x)$ mit $k\geq 1$\\%umgekehrtes define
f ist streng monoton wachsend/steigend 
%bild tafel 2.2
\item{$h: \R \to \R, h(x) = [x]$}\\
% Stufenfunktion Diagramm
h ist monoton wachsend, aber nicht streng monoton.\\
Monoton wachsend: $x < y \Rarr [x] < [y]$\\
$x < y \not\Rarr [x] < [y]$\\
z. B.: $1,2 < 1,3 , [1,2] = 1 = [1,3] $}
\item{Exponentialfunktion\\
$exp: \R \to \R, exp(x)= \ds\sum_{k=0}^{\infty} \frac{x^k}{k!}$\\
Ist streng monoton wachsend.\\
\bew
\begin{enumerate}
\item{$exp(0) = 1 + \frac{0}{1!} + \frac{0}{2!} + ... = 1$}
\item{Sei $a > 0$\\
$exp(a) = = 1 + \frac{a}{1!} + \frac{a}{2!} + ... > 1$}
\item{Sei $a > 0 exp(-a) \cdot exp(a) = exp(-a + a) = exp(0) = 1$\\
$\Rarr exp(-a) = \frac{1}{exp(a)} \Rarr 0 < exp(a) < 1$\\
%das kann ich nicht lesen... (iPhone Bild)
$exp(b) > 0$ für alle $b \in \R$}
\item{Sei $a > b$\\
$exp(a) = exp(a - b + b) = exp(a - b) \cdot exp(b)$ %overbraces über exp(a - b) => {> 1} über exp(b) => {> 0}
> exp(b) $\Rarr$ exp streng monoton wachsend \qed }
\end{enumerate}
}
\end{enumerate}
%
\chapter{Stetigkeit}
\underline{Idee:} Eine Funktion ohne sprünge heißt \underline{stetig}\\
\sS{Definition}
Sei $D\subseteq \R,\ f:D→\R$ eine Funktion\\
\begin{enumerate}
\item{f heißt stetig in $x\in D$ wenn gilt:\\
Für jedes $\e>0$ gibt es ein $\delta>0$ so dass für jedes $y\in D$ mit $|x-y|<\delta$ gilt $|f(x)-f(y)<\e$ %graph 6.2
}
\item{f heißt stetig wenn f in jedem $x\in D$ stetig ist}
\end{enumerate}
\sS{Beispiel}
\begin{enumerate}
\item{Die Funktion $id:\R→\R,\ x\mapsto x$ ist stetig}
\item{Die Funktion $f:\R→\R,\ f(x)=x^2$ ist stetig. %graph klein
\bew
Sei $x,y\in\R\quad y=x+k$.\\
$$f(y)-f(x)=(x+h)^2-x^2=x^2+2xh+h^2-x^2=2xh+h^2$$\\
Wähle jedenfalls $\delta\leq 1$. Wenn $|h|=|x-y|>\delta$ dann $|h|<1$\\
$$|f(y)-f(x)|=|2xh+h^2|\leq|2x|·|h|+|h|^2<|2x|·|h|+|h|=(|2x|+1)·|h|$$\\
Gegeben sei $\e>0$\\
Wähle $\delta=min\left\{1,\dfrac{\e}{|2x|+1}\right\}$\\
Wenn $|x-y|<\delta$ dann $$|f(x)-f(y)|<(2|x|+1)·|h|<(2|x|+1)·\dfrac{\e}{2|x|+1}=\e$$\\
Also $f$ stetig in $x$}
\item{$g:=\R \to \R,\ g(x):=\{x\}$\\
%graph
g ist stetig an $x$ \equ $x\notin\Z$
\Bew{g nicht stetig an $x\in\Z$:}
Zeige: es gibt ein $\e>0$ so dass kein $\delta>0$ existiert mit: $|x-y|>\delta\Rarr|g(x)-g(y)|<\e$\\
z.B. $\e=1$ Sei $\delta>0.\ y=x-\dfrac{\delta}{2}\quad |x-y|=\dfrac{\delta}{2}<\delta$\\
aber $g(y)=\{x-\dfrac{\delta}{2}\}=x-1$ (weil $x\in\Z$)\\
$|g(x)-g(y)|=|x-(x-1)|=1 \not<\e\qed$}
\end{enumerate}
%
\sS{Satz}
Die Exponentialfunktion $exp: \R \to \R$ ist stetig.\\
\bew
Verwende nur:
\begin{itemize}
\item{Funktionalgleichung: $exp(x + y) = exp(x) \cdot exp(y)$}
\item{exp ist streng monoton wachsend}
\item{exp(0) = 1}
\end{itemize}
\subsection*{Behauptung}
Für jedes $\epsilon > 0$ gibt es ein $n \in \N$ mit $exp(\frac{1}{n}) < 1 + \epsilon$\\
%Graf
Angenommen, $exp(\frac{1}{n}) \geq 1 + \epsilon$\\
Dann $exp(1) = \frac{1}{n} + ... \frac{1}{n}$ % underbrace unter den Brüchen {n}
\phantom{Dann $exp(1) $} $= exp(\frac{1}{n}) + ... + exp(\frac{1}{n}) = exp(\frac{1}{n})^n$ \\
\phantom{Dann $exp(1) $} $ \geq (1 + \epsilon)^n \geq 1 + n \epsilon $ %(Pfeil auf letztes geq Bernoulli)
\\
$exp(1) \geq 1 + n \epsilon$\\
$n \leq \frac{exp(1) - 1}{\epsilon}$\\
Das gilt nur für endliche viele $n \in \N$\\
$\Rarr$ Beh.\\
\underline{Zeige:} exp ist stetig an 0. Gegeben sei $\epsilon > 0, OE ? \epsilon < 1$\\
Wähle $n \in \N$ mit $exp(\frac{1}{n}) < 1 + \epsilon$\\
$$\Rarr exp(-\frac{1}{n}) = exp(\frac{1}{n})^{-1} < \frac{1}{1 + \epsilon} = \frac{1 - \epsilon}{(1+\epsilon)(1-\epsilon)} = \frac{1-\epsilon}{1 - \epsilon^2} > 1 - \epsilon$$\\
Sei $\delta \frac{1}{n}$\\
Sei $y \in \R, |0 - y| < \delta = \frac{1}{n}$\\
$|y| < \frac{1}{n}$ d.h.\\
$-\frac{1}{n} < y < \frac{1}{n}$\\ \\
exp streng monoton wachsend.\\
$$\Rarr 1 - \epsilon < exp(-\frac{1}{n}) < \exp(y) < exp(\frac{1}{n}) < 1 + \epsilon$$\\
$\Rarr |exp(y) - exp(0)| < \epsilon$ Also exp stetig in 0\\
\underline{Zeige:} exp ist eine stetig in $x \in \R$. Gegeben sei $\epsilon > $\\
Sei $y = x + h$, $|h| < \delta$ ($\delta$ noch zu wählen)
$$|exp(y) - exp(x)| = |exp(x + h) - exp(x)| = |exp(x) \cdot exp(h) - exp(x)| = exp(x) \cdot exo(h) -1$$\\
$|exp(y) -exp(x) | < \epsilon$\\
$$\Leftrightarrow exp(x) \cdot |exp(h) - 1| < \epsilon \Leftrightarrow exp(h) - 1 < \frac{\epsilon}{exp(x)} = \epsilon '$$\\
Weil exp stetig in 0 ist gibt es ein $\delta > 0$ mit $|h| < \delta \Rarr |exp(h) -1| < \frac{\epsilon}{exp(x)}$\\
$\Rarr$ exp ist stetig in x \qed\\
%
\sS{Satz (Folgenstetigkeit)}
Sei $D\subseteq\R,\ x\in D,\ f:D→\R$ Funktion $f$ ist genau dann stetig in $x$ wenn gilt:\\
\begin{itemize}
\item{Für jede Folge $(x_n)_{n\geq 0}$ mit $x_n\in D,\ x_n→x$ für $n→∞$ gilt auch $f(x_n)→f(x)$ für $n→∞$}
\end{itemize}
%
\sS{Satz}
Sei $D\subseteq\R,\ f,g:D→\R$ in $x\in D$\\
Dann gilt:\\
\begin{itemize}
\item{$f+g:D→\R$ stetig in $x$}
\item{$f·g:D→\R$ stetig in $x$}
\item{Wenn $g(x)\neq 0$ für alle $x'\in D$}
\end{itemize}
Dann ist $\dfrac{1}{f}:D→\R$ stetig in x.
\Bew{mit Folgenstetigkeit}
Sei $x_n \to x$ für $n \to \infty$\\
mit $x_n \in D$\\
$f, g$ stetig $Rarr f(x_n) \to f(x)$ \\
\phantom{$f, g$ stetig $Rarr$} $g(x_n) \to g(x)$\\
$\Rarr f(x_n) + g(x_n) \to f(x) + g(x)$
$\phantom{\Rarr} f(x_n) \cdot g(x_n) \to f(x) \cdot g(x)$\\
Wenn also $f(x) \neq 0$\\
$f(x_n)^{-1} \to f(x)^{-1}$\\
$\Rarr f + g, f) \cdot g, \frac{1}{f}$ stetig in x \qed
% Kopfzeile beim Kapitelanfang:
\fancypagestyle{plain}{
%Kopfzeile links bzw. innen
\fancyhead[L]{\calligra\Large Vorlesung Nr. 8}
%Kopfzeile rechts bzw. außen
\fancyhead[R]{\calligra\Large 05.11.2012}
}
%Kopfzeile links bzw. innen
\fancyhead[L]{\calligra\Large Vorlesung Nr. 8}
%Kopfzeile rechts bzw. außen
\fancyhead[R]{\calligra\Large 05.11.2012}
% **************************************************
%
\Wdh{Konvergenzsätze}
\begin{itemize}
    \item{Eine monoton wachsende und beschränkte Folge konvergiert zwangsläufig.}
    \item{Eine Reihe $\sum_{k=0}^{∞} a_k$ mit $a_k \geq 0$ für alle $k$ konvergiert \equ die Folge der Partialsummen $(S_n = \sum_{k=0}^{n} a_k)_{n \in \N}$ ist beschränkt}
\end{itemize}
%
\bsp
    $\sum_{k=0}^{n} \frac{1}{k} = 1 + \frac{1}{2} + \frac{1}{3} …$ ist unbeschränkt
\bsp
    $\sum_{k=1}^{\infty} \frac{1}{k^2} = 1 + \frac{1}{4} + \frac{1}{9}+…$
\sss{Leibnitz} 
    Sei $(a_n)$ monoton fallende Nullfolge.\\
    Dann konvergiert $\sum_{k=0}^{∞} (-1)^k \cdot a_k$
\bsp
    $(1 - \frac{1}{2} + \frac{1}{3} - \frac{1}{4})$ … konvergiert.
% satz 4.10 ***************
%\setcounter{chapter}{4}
%\setcounter{section}{9}{Die Folge }
% *************************

\sS{Satz Verdichtungslemma von Cauchy}
Sei $(a_n)$ monoton fallende Nullfolge.\\*
Die Reihe $\Sum_{k=0}^{\infty} a_k$ konvergiert genau dann, wenn die verdichtete Reihe $\Sum_{k=0}^{\infty} 2^k \cdot a_{2^k} = 1 · a_1 + 2 · a_2 + 4 \cdot a_4$ … konvergiert.
%
\bsp
    $a_k = \frac{1}{k}\qquad (k \geq 1)$\\*
    $2^k \cdot a_{2^k} = 2^k \cdot \frac{1}{2^k} = 1$

\section*{Satz} % Keine Nummerierung!
$\Sum_{k=0}^\infty \frac{1}{k}$ konvergent \equ\ $\Sum_{k=0}^{∞} 1$ konvergent (ist nicht der Fall.)
\bew
Sei $b_n = \ds\sum_{k=2^n}^{2^{n+1}-1} a_k$\\*
Für $\ds 2^n \leq k \leq 2^{n+1} - 1$ ist $\ds a_{2^n} \geq a_{k} \geq a_{2^{n+1} - 1} \geq a_{2^{n+1}}$ \Rarr $\ds 2^n · a_{2^n} \geq b_n \geq  2^n · a_{2^{n+1}}$\\*
Wenn $\ds \sum_{k \geq 0} 2^k · a_2^k$ beschränkt \Rarr $\ds \sum_{k \geq 0} b_k$ beschränkt \Rarr $\ds \sum_{k \geq 0} a_k$ beschränkt\\*
Hier immer beschränkt \equ konvergent\\*
Wenn $\ds \sum_{k \geq 0} 2^k · a_k$ beschränkt \Rarr $\ds \sum_{k \geq 0} b_k$ beschränkt \Rarr $\ds \sum_{k \geq 0} 2^{k} · a_2^{k+1}$ beschränkt \equ $\ds \sum_{k \geq 0} 2^{k+1} · a_2^{k+1}$ beschränkt \equ $\ds \sum_{k \geq 0} 2^{k} · a_2^{k}$ beschränkt\\*
Das zeigt den Satz.\\*
\sss{Anwendung}
\sss{Erinnerung}
    Für $ x\geq 0$ und $a \in \R$ wird später $x^a \in R$ definiert\\*
    Wenn $\ds a = \frac{n}{m}$ mit $m \geq 1$ d.h. $a \in \Q$ dann $\ds x^a = \sqrt[m]{x^n}$.\\*
    Wenn $x > 1$ dann gilt: \\*
    $x^a = \begin{cases} >1 \text{ wenn }a>0\\* =1 \text{ wenn }a=0\\* <1\text{ wenn }a<0 \end{cases}$

\sS{Satz}
Sei $a \in \R$. Die Reihe $\ds \sum_{k=1}^{∞} \frac{1}{k^a}$ konvergiert genau dann, wenn $a > 1$
\bew
Wenn $a \leq 0$ dann $\frac{1}{k^a} \geq 1$ \Rarr Reihe divergiert.\\*
Sei $a >0$, sei $a_n = \frac{1}{n^a}$ \\*
$\ds n < n+1 \Rarr n^a < (n+1)^a \Rarr a_n > a_{n+1}$ Somit $(a_n)$ monoton fallend.\\*
$\ds \lim_{n\to \infty} n^a = \infty \Rarr \lim_{n \to \infty} \frac{1}{n^a} = 0$ \Rarr Verdichtungslemma ist anwendbar.\\*
Bilde $\ds 2^n · a_{2^n} = 2^n · \frac{1}{(2^n)^a} = 2^n · 2^{-n · a} = 2^{n(1-a)} = (2^{1-a})^n = x^n$\\*
mit $x:=2^{1-a}$\\*
Erhalte:
$\ds \sum_{k=1}^\infty \frac{1}{k^a}$ konvergiert $\equ\ \ds \sum_{k=0}^\infty x^k$ konvergiert $\equ |x|<1 \equ x<1 \equ 2^{1-a} <1$\\*
$\equ 1-a<0 \equ a>1$\qed\\*[4pt]
Beziehung:
$\ds \sum_{k=1}^\infty\frac{1}{k^a}=\zeta (a)$ für $a>1$\\*
Riemannsche Zetafunktion
Spezielle Werte:
$\zeta (2) = \sum_{k\geq1}\frac{1}{k^2}=\frac{\pi^2}{6}$\\*
$\zeta (4) = \sum_{k\geq1}\frac{1}{k^2}=\frac{\pi^4}{90}$\\*
$\zeta (6) = \sum_{k\geq1}\frac{1}{k^2}=\frac{\pi^6}{945}$\\*
Frage: Für welche z ist $\zeta(z)=0$?

\uS{Teilfolgen}
\sS{Definition Teilfolge}
Sei $(a_n)$ eine Folge reeller Zahlen.\\*
Eine Teilfolge von $(a_n)$ ist eine Folge der Form $(a_{n_k})_{k \geq 0}$ wobei $n_0, n_1, n_2,…$ streng monoton wachsende Folge in $\N_0$ ist.\\*
\bsp
$(a_n) = (1, x, x^2, x^3 , x^4 …)$\\*
$(n_k) = (1, 4, 9, 16) \leadsto $ Teilfolge $(x, x^4, x^9, x^{16} ,…)$\\*

\sS{Bemerkung}
Wenn $a_n \to a$ für alle $n \to \infty$ dann konvergiert jede Teilfolge von $(a_n)$ gegen $a$ (Präsenzübung Nr. 9)

\uS{Schlüsselsatz}
\sS{Lemma}
Jede Folge reeller Zahlen $(a_n)_{n\geq 0}$ hat eine monotone Teilfolge.\\*
%
\bew
Wir nennen $n\in \N_0$ \underline{extrem} wenn $a_n \geq a_m$ für alle $m \geq n$\\*
Unterscheide zwei Fälle:\\*
\begin{itemize}
    \item{Es gibt unendlich viele extreme $n \in \N$\\*
Dies seien $n_0, < n_1, n_2…$\\*
Dann $a_{n_0} \geq a_{n_1} \geq a_{n_2} …$\\*
Weil $n_0$ extrem … weil $n_1$ extrem.\\*
→ monoton fallende Teilfolge gefunden}
    \item{Es gibt nur endlich viele extreme $n$\\*
Wähle $n_0 \in \N$ s.d. gilt: $m \geq n_0\ \Rarr\ m$ nicht extrem.\\*
$n_0$ nicht extrem $\Rarr$ es gibt $n_1 \geq n_0$ mit $a_{n_1} > a_{n_0}$ insbesondere $n_1 > n_0$\\*
$n_1$ \phantom{nicht extrem }$\Rarr$ \phantom{es gibt }$n_2 \geq n_1$ mit $a_{n_2} > a_{n_1}$ insbesondere $n_2 > n_1$\\*
$n_2$ \phantom{nicht extrem }$\Rarr$ \phantom{es gibt }$n_3 \geq n_2$ mit $a_{n_3} > a_{n_2}$ insbesondere $n_3 > n_2$\\*
usw.\\*
Erhalte $n_0 < n_1 < n_3 < …$ mit $a_{n_0} < a_{n_1} < a_{n_2} < …$ \\*
$\to $ streng monoton wachsende Teilfolge gefunden.\qed
}
\end{itemize}

\sS{Satz Bolzano-Weierstraß}
Jede beschränkte Folge reeller Zahlen hat eine konvergernte Teilfolge.
\bew
Es gibt ein monotone Teilfolge (Lemma 4.14)\\*
Diese ist beschränkt \Rarr konvergent.

\sS{Definition Cauchyfolge}
Eine Folge reeller Zahlen $(a_n)_{n \geq 0}$ heißt Cauchyfolge wenn gilt:\\*
Für jedes $\e > 0$ gibt es ein $N \in \N$ sodass für $m, n \geq N$ gilt: $|a_n - a_m| < \e$

\sS{Satz}
Eine Folge reeller Zahlen $(a_n)$ konvergiert genau dann, wenn sie eine Cauchyfolge ist.
\bew
\begin{description}
\item["\Rarr"]{Sei $a_n \to a$ für $n \to \infty$\\*
Gegeben sei $\e > 0$. Es gilt $N \in \N$ so dass $|a_n - a| < \frac{\e}{2}$ für $n \geq N$\\*
Für $n, m \geq N$ gilt:\\*
$|a_n - a_m| = |a_n - a + a - a_m| \leq |a_n - a| + |a - a_m| < \frac{\e}{2} + \frac{\e}{2} = \e$\\*
$\Rarr (a_n)$ ist eine Cauchyfolge\\*}
\item["\Larr"]{Sei $(a_n)$ eine Cauchyfolge
\sss{Behauptung} $(a_n)$ ist beschränkt
\bew
Wähle $\e=1$ Es gibt $N\in\N$ mit $|a_n-a_m|<1$ für $m,n\geq N$\\*
Sei $C=max\{|a_0|,|a_1|,|a_2| … |a_N|,|a_N|+1\}$\\*
Dann $|a_n| \leq C$ für alle $\N$\\*
$(n\geq N \Rarr |a_n-a_N| < 1 \Rarr |a_n|<|a_N|+1)$\\*
Also ist $(a_n)$ beschränkt\\*
$\underset{Lemma}{\Rarr}\ (a_n)$ hat eine monotone Teilfolge $(a_{n_k})_{k\geq0}$ diese ist beschränkt \Rarr konvergent.\\*
Sei $\ds \lim_{k→∞}(a_{n_k})$
\sss{Behauptung}
$a_n→a$ für $n→∞$\\*
Sei $\e>0$ gegeben. Es gibt $n\in\N$ so dass
\begin{enumerate}
\item{$n,m\geq N \Rarr |a_n-a_m|< \frac{\e}{2}$}
\item{$k\geq N \Rarr |a_{n_k}-a|< \frac{\e}{2}$}
\end{enumerate}
Sei $k\geq N$
\bem
Für jedes $k\in\N$ ist $n_k\geq k$\\*
$\ds |a_k-a|=|a_{n_k}+a_{n_k}-a| \leq |a_k-a_{n_k}|+|a_{n_k}-a|<\frac{\e}{2}+\frac{\e}{2}=\e$\\*
Also $a_k→a$ für $n→∞$\qed}
\end{description}

\uS{Umformulierung für Reihen}
\sS{Satz (Cauchy-Kriterium für Reihen)}
Eine reelle Reihe $\ds \sum_{k=0}^\infty a_k$ konvergiert genau dann, wenn gilt:\\*
Für jedes $\e>0$ gibt es ein $N\in\N$ so dass für alle $\ds n,m\geq N,\ n\leq m$ $$\left|\sum_{k=n}^m\right|<\e$$
%
\sss{Beweis: Partialsummen}
$\ds s_n=\sum_{k=0}^n a_k$\\*
$\ds \sum_{k=n}^m=s_m-s_{n-1}$\\*
Damit ist 4.19 äquivalent zu 4.18
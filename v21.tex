% Kopfzeile beim Kapitelanfang:
\fancypagestyle{plain}{
%Kopfzeile links bzw. innen
\fancyhead[L]{\calligra\Large Vorlesung Nr. 21}
%Kopfzeile rechts bzw. außen
\fancyhead[R]{\calligra\Large 07.01.2013}
}
%Kopfzeile links bzw. innen
\fancyhead[L]{\calligra\Large Vorlesung Nr. 21}
%Kopfzeile rechts bzw. außen
\fancyhead[R]{\calligra\Large 07.01.2013}
% **************************************************
%
\wdh
Eine Funktion $f: I \to \R$ heißt differenzierbar in $x_0 \in I$ wenn der Limes
$$f(x) = \lim_{x\to x_0} \frac{f(x) - f(x0)}{x-x_0}\text{ existiert}$$
Ableitungsregeln: Produkt, Kettenregel, Umkehrfunktion $\leadsto$ Kann "alle" Ableitungen ausrechnen
8.11 $f: I = (a,b) \to \R$ differenzierbar, $f$ hat ein lokales extremum in $x_0 \in (a, b)$ \Larr{} $f'(x) = 0$\\*
\begin{tikzpicture}[domain=-0.3:2,prefix=plots/, smooth]
\draw[very thin,color=gray] (-0.3,-0.25) grid (1.99,1.99);
\draw[->] (-0.3,0) -- (2.5,0) node[right] {$x$};
\draw[->] (0,-0.3) -- (0,2) node[above] {$y$};
% Graphen beschriftung
\draw[color=blue] plot[id=21.1_cosp1] function{cos(x) + 1} node[below, midway] {};
\end{tikzpicture}\\*
8.12 (Satz von Rolle)\\*
Sei $f: [a, b] \to \R$  diff'bar\\*
\begin{tikzpicture}[domain=-0.3:2,prefix=plots/, smooth]
\draw[very thin,color=gray] (-0.3,-0.25) grid (1.99,1.99);
\draw[->] (-0.3,0) -- (2.5,0) node[right] {$x$};
\draw[->] (0,-0.3) -- (0,2) node[above] {$y$};
% Graphen beschriftung
\draw[color=blue] plot[id=21.2_cos_tiefer] function{cos(x) * 1.5 + 1};
\end{tikzpicture}
\\*
\begin{tikzpicture}[domain=-0.3:2,prefix=plots/, smooth]
\draw[very thin,color=gray] (-0.3,-0.25) grid (1.99,1.99);
\draw[->] (-0.3,0) -- (2.5,0) node[right] {$x$};
\draw[->] (0,-0.3) -- (0,2) node[above] {$y$};
% Graphen beschriftung
% Graph ausdenken, oder als Linie zeichnen.
\end{tikzpicture}
$f(a) = f(b)$ dann existiert $x_0 \in (a, b)$ mit $f'(x) = 0$

\sS{Satz (Mittelwertsatz der Differenzialrechnung)}
Sei $f:[a,b]→\R$ stetig, auf (a,b) differenzierbar, dann gibt es ein $x_0\in (a,b)$ mit.
$$f'(x_0)=\frac{f(b)-f(a)}{b-a}=\lambda$$
GRAPH
\bew
Sei $g:(a,b)→\R,\ g(x)=f(x)-\lambda·x$
$$\text{Rechne }g(a)-g(b)=f(a)-\lambda·a-f(b)+\lambda·b = f(a)-f(b)-\lambda(a-b)=f(a)-f(b)-\frac{f(b)-f(a)}{b-a}(a-b)=0$$
Satz von Rolle auf $g$ anwendbar \Rarr\ es gibt $x_0\in (a,b),\ g'(x_0)=0$
$$f(x)=g(x)+\lambda x\ \Rarr\ f'(x_0)=g'(x_0)+\lambda=\lambda$$ \qed

\sS{Folge}
Sei $f: T \to \R$ diffbar, $f'(x) = 0$ für alle $x$ dann ist $f$ konstant.
\bew
Sei $x_1 < x_2$ in $I$\\*
Es gilt $x_0$ mit $x_1 < x_0 < x_2$ mit $f(x_1) - f(x_2) = f(x_0) \cdot f(x_1 - x_2) = 0$\\*
\Rarr{} $f(x_1) = f(x_2) \Rarr f$ konstant. \qed{}\\*
Mittelwertsatz \qed

\sS{Satz (Monotonie)}
Sei $f:[a,b]→\R$ stetig, diff'bar auf $(a,b)$\\*
$f'(x)\geq 0$ für alle $x\in (a,b)$ \equ\ $f$ monoton wachsend
$f'(x)\leq 0$ für alle $x\in (a,b)$ \equ\ $f$ monoton fallend
$f'(x)> 0$ für alle $x\in (a,b)$ \Rarr\ $f$ streng monoton wachsend
$f'(x)< 0$ für alle $x\in (a,b)$ \Rarr\ $f$ streng monoton fallend
\bew
Angenommen $f'(x)\geq 0$ für alle $x$\\*
Gegeben sei $a<x_1<x_2<b$
\sss{Zeige}
$f(x_1)\leq f(x_2)$\\*
Mittelwertsatz: es gibt $x_0$ mit $x_1<x_0<x_2$ und $f(x_2)-f(x_1)=\underbrace{f'(x_0)}_{\geq 0}\underbrace{x_2-x_1}_{>0}$ \Rarr\ $f(x_2)\geq f(x_1)$, also monoton wachsend.\\*
Analog folgen alle "\Rarr" des Satzes.\\*
Angenommen $f$ monoton wachsend\\*
Sei $x_0 \in (a, b)$\\*
Zeige: $f'(x) \geq 0$\\*
% TOFIX Richtigen Arrow suchen
$f'(x_0) = \lim_{x \searrow x_0}\frac{f(x) - f(x_0)}{x - x_0}$\\*
$x > x_0 \Rarr x - x_0 > 0,\ f(x) - f(x_0) \geq 0$\\*
Analog: für monoton fallend \Rarr\ $f'(x)\leq 0$ für alle $x$\qed
\bsp
\enum{
	\item{$cos:[0,\pi]→\R$ streng monoton fallend\\*
\begin{tikzpicture}[domain=-0.3:3.5,prefix=plots/, smooth]
\draw[very thin,color=gray] (-0.3,-1.25) grid (3.49,1.25);
\draw[->] (-0.3,0) -- (3.5,0) node[right] {$x$};
\draw[->] (0,-1.25) -- (0,1.5) node[above] {$y$};
% Graphenbeschriftung bei PI
\draw (3.14,0) node[anchor=north] {$\pi$};
\draw[color=blue] plot[id=21.4_cos] function{cos(x)};
\end{tikzpicture}
\begin{tikzpicture}[domain=-0.3:3.5,prefix=plots/, smooth]
\draw[very thin,color=gray] (-0.3,-1.25) grid (3.49,1.25);
\draw[->] (-0.3,0) -- (3.5,0) node[right] {$x$};
\draw[->] (0,-1.25) -- (0,1.25) node[above] {$y$};
% Graphenbeschriftung bei PI
\draw (3.14,0) node[anchor=north] {$\pi$};
\draw[color=blue] plot[id=21.5_sin] function{sin(x)};
\end{tikzpicture}\\*
	$cos'=-sin,\ -sin(x)<0$ für alle $x\in (0,\pi)$.
	}
	\item{$f:\R→\R,\ f(x)=x^3$\\*
\begin{tikzpicture}[domain=-1.5:1.5,prefix=plots/, smooth]
\clip (-1.5,-4) rectangle (1.5,3.5);
\draw[very thin,color=gray] (-1.49,-3.99) grid (1.49,3.24);
\draw[->] (-1.6,0) -- (1.6,0) node[right] {$x$};
\draw[->] (0,-3) -- (0,3) node[above] {$y$};
\draw[color=blue] plot[id=21.4_x3] function{x**3} (0.5, -2.25) node {$f(x) = x^3$};
\draw[color=red] plot[id=21.5_3x2] function{3*(x**2)} (0, 2.25)node {$f'(x) = 3x^2$};
\end{tikzpicture}\\*
	$f'(x)\geq 0$ für alle $x$ $f'(0)=0$ trotzdem $f$ streng monoton wachsend
	}
}

\sS{Satz}
Sei $f:(a,b)→\Re$ in $x_0$ zweimal differenzierbar mit $f'(x_0)=0$, dann gilt:
\enum{
	\item{Wenn $f''(x_0)<0$ dann hat $f$ in $x_0$ ein lokales Maximum}
	\item{Wenn $f''(x_0)>0$ dann hat $f$ in $x_0$ ein lokales Minimum}
}
(Wenn $f''(x_0=0)$, dann keine Aussage)
\bew
Sei $f''(x_0)<0)$.
$$f''(x_0)=\lim_{x→∞}\frac{f'(x)-f'(x_0)}{x-x_0}$$
\Rarr Es gibt ein $ε>0$, so dass 
$$|x-x_0|<ε\ \Rarr\ \frac{f'(x)-f'(x_0)}{x-x_0}<0$$
% rest bis pause
d.h.
\begin{itemize}
\item[a]{$x_0 < x < x_0 + \e$ \Rarr{} $f'(x) - f'(x_0) < 0$ \Rarr $f'(x) < 0$}
\item[b]{$x_0 - \e < x < x_0$ \Rarr{} $f'(x) - f'(x_0) > 0$ \Rarr $f'(x) > 0$}
\end{itemize}
8.15 \Rarr{} $f$ streng monoton fallend auf $[x_0, x_0 + \e]$ wegen a)
			$f$ streng monoton steigend auf $[x_0 - \e, x_0]$ wegen b)
\bsp $f(x) x^3 - 3x$ $f: \R \to \R$\\*
$f'(x) = 3x^2 - 3$\\*
$f''(x) = 6x$\\*
Nullstelle (NST) von $f'$: $f'(x) = 0 \equ 3x^2 - 3 = 0 \equ x^2 = 1 \equ x \in \{1, -1\}$\\*
Anwendung von $f''$ an NST von $f'$: $f(1) = 6$

\uS{Regeln von de l' Hospital}
Ziel: Berechnung eines Limes $\lim_{x→a}\frac{f(x)}{g(x)}$ wenn $\lim_{x→a} f(x)=0=\lim_{x→a} g(x)$ oder $\lim_{x→a}g(x)=\pm ∞$

\sS{Satz}
Sei $I=(a,b)$ mit $-∞\leq a<b\leq ∞$\\*
Seien $f,g:I→\R$ differenzierbare Funktionen
\sss{Annahme}
Der Limes
$$\lim_{x→a}\frac{f'(x)}{g'(x)}=c\eR\text{ existiert}$$
\enum{
	\item{Wenn $\lim_{x→a}f(x)=\lim_{x→a}g(x)=0$, dann gilt $\lim_{x→a}\frac{f(x)}{g(x)}=c$}
	\item{Wenn $\lim_{x→a}g(x)=∞$ oder $-∞$, dann $\lim_{x→a}\frac{f(x)}{g(x)}=c$}
}
Analog für $x \to b$ (ohne Beweis)
\bsp
\begin{enumerate}
\item{$\lim_{x\to 0} \frac{sin(x)}{x} = ?$ \\*
$\lim_{x\to 0} x = 0$, $\lim_{x\to 0} \frac{sin(x)} = 0$\\*
$x' = 1, sin' = cos$\\*
$\leadsto$ Berechne\\*
$\lim_{x\to0} \frac{cos(x)}{1} = cos(0) = 1$ existiert.\\*
l'Hospital \Rarr{} $lim_{x \to 0} \frac{sin(x)}{x} = 1$}
\item{$\lim_{x \to \infty} \frac{log(x)}{x}$\\*
$\lim_{x \to \infty} x = \infty$\\*
$log(x)' = \frac{1}{x}$, $x' = 1$\\*
$\leadsto$ Berechne $$\lim_{x \to \infty} \frac{\frac{1}{x}}{1} = \lim_{x \to \infty} \frac{1}{x} = 0$$
8.17 \Rarr $\lim_{x \to \infty} \frac{log(x)}{x} = 0$}
\item{Rationale Funktion\\*
$\lim_{x \to \infty} \frac{x^3 + x}{2x^2 + 5}$\\*
$f(x) = x^3 + x$, $g(x) = 2x^2 + 5$\\*
$\lim_{x \to \infty} g(x) = \infty$\\*
$\leadsto \ f'(x) = 2x + 1$, $g(x) = 4x$\\*
$\leadsto$ Rechne\\*
$\lim_{x \to \infty} \frac{2x + 1}{4x}$\\*
$= \lim_{x \to \infty} (\frac{1}{2} + \frac{1}{4x}) = \frac{1}{2}$
$\Rarr \lim_{x \to \infty} \frac{x^2 + x}{2x^2 + 5} = \frac{1}{2}$
}
\item{$$\lim_{x→0}\left(\frac{1}{sin(x)}-\frac{1}{x}\right)$$
$$\frac{1}{sin(x)}-\frac{1}{x}=\frac{x-sin(x)}{x·sin(x)}=\frac{f(x)}{g(x)}$$
$$f(x)=x-sin(x),\ g(x)=x·sin(x)$$
$$\lim_{x→0}\left(x-sin(x)\right)=0=\lim_{x→0}\left(x·sin(x)\right)$$
$$f'(x)=1-cos(x),\ g'(x)=sin(x)+x·cos(x)$$
$$\lim_{x→0}\frac{1-cos(x)}{sin(x)+x·cos(x)}=?$$
$$\lim_{x→0}(1-cos(x))=0=\lim_{x→0}sin(x)+x·cos(x)$$
Wende 8.17 nochmal an
$$f''(x)=sin(x),\ g''(x)=cos(x)+cos(x)-x.sin(x)$$
$$lim_{x→0}\frac{f''(x)}{g''(x)}=lim_{x→0}\frac{sin(x)}{2cos(x)-x·sin(x)}=\footnote{$lim_{x→0}2cos(x)-x·sin(x)=2$\\*
$lim_{x→0}sin(x)=0$}\frac{0}{2}=0$$
$$\underset{8.17}{\Rarr} lim_{x→0}\frac{f'(x)}{g'(x)=0}\underset{8.17}{\Rarr} lim_{x→0}\frac{f(x)}{g(x)=0}$$
}
\end{enumerate}
%Kopfzeile links bzw. innen
\fancyhead[L]{\calligra {\Large Vorlesung Nr. 12}}
%Kopfzeile rechts bzw. außen
\fancyhead[R]{\calligra \Large{19.11.2012}}
% **************************************************
\wdh
Sei $D \subseteq \R$. eine Funktion $f: D \to \R$ ist stetig in $x \in D$ wenn gilt:\\
\begin{array}{ll}
\text{Für jedes $\epsilon > 0$ gibt es ein $\delta > 0$ so dass gilt:}\\
\text{wenn $y \in D$ mit $|x - y| < \d$ dann $|f(x) - f(y)| < \e$}
\end{array}
%Stefan
%Diagramm, Gausklammern
\sS{Satz Folgenstetigkeit}
Eine Funktion $f: D \to \R$ ist stetig in $x \in D$ $\equ$ Für jede Folge $(x_n)$ mit $x_n \in D$ für alle $n$ und $x_n \to x$ gilt auch $f(x_n) \to f(x)$.\\
(d.h. f erhält Konvergenz)\\
%Stefan
\wdh
\sS{Satz 6.5}
Wenn $f, g: D \to \R$ stetig in $x \in D$ dann auch $f + g,\ f \cdot g$ und $\frac{1}{g}$ (falls $g(y) \neq 0$ für alle $y \in D$) \\
Und $a \cdot f$ für $a \in \R$
\sS{Korollar} 
Polynomfunktionen und rationale Funktionen sind stetig.
%Stefan
\sS{Satz Stetigkeit der Komposition}
Sei $f: C \to \R$, $g: D \to \R$ Funktionen mit $f(C) \subseteq D$. Wenn $f$ stetig in $x \in D$ und g stetig in $f(x)$ dann ist:\\
$g \cird f: C \to \R$ stetig in x.\\
\bew mit Folgenstetigkeit:
Sei $x_n \to x$ mit $x_n \in C$\\
$f$ stetig in $x \Rarr \ f(x_n) \to f(x)$\\
$g$ stetig in $f(x) \Rarr g(f(x_n)) \to g(f(x))$\\
d.h. $(g \circ f)(x_n) \to (g \circ f)(x)$\\
also ist $g \circ f: C \to \R$ stetig in $x$ \qed
%Stefan
\ul{Vorher} $a = 0$ ist Berührpunkt vpn $D$ und $D \cap (0, \infty)$ und $D \cap (- \infty , 0)$, denn $\frac{1}{n} \to 0$ und $-\frac{1}{n} \to 0$\\
Sei $g: \R \to \R$, $g(x) = x^3$\\
$\infty , - \infty$ sind Berührpunkte von $D = \R$\\
$\ds\lim_{x \to \infty} g(x) = \infty$\\
$\ds\lim_{x \to -\infty} g(x) = -\infty$\\ 
\bem{Umformulierung von Satz 6.4}
Eine Funktion $f: D \to \R$ ist stetig in $a \in D \\ $
$\equ \quad \ds\lim_{x \to a} f(x) = f(a)$
%Stefan
\sS{Definition}
Sei $M \in \R$ eine nicht-leere Teilmenge wenn $M$ nach oben unbeschränkt, schreibe $sub(M) = \infty$ (Sprich: uneigentliches Supremum)\\
%Stefan
\bsp
\begin{enumerate}
\item{$f: (0, 1) \to \R, \quad f(x) = \frac{1}{x}$ % Graph
$\ds\lim_{x \to 0} f(x) = \infty$\\
Somit $f$ nicht nach oben beschränkt}
\item{$g(0,1) \to \R, \ g(x) = x$\\
$g$ beschränkt. $g((0,1)) = (0,1)$\\
$sup \{\g(x) | x \in (0, 1) \} = 1$\\
Aber $g(x) < 1$ für alle $x \in (0, 1)$. Also nimmt g nicht ihr Maximum an.}
\end{enumerate}
%Stefan
Und dann gilt: &$f(x_n_k) \to y$ für $k \to \infty$ (Teilfolge einer Konvergenten Folge)\\
			   &$f(x_n_k) \to y$ für $k \to \infty$ weil $f$ stetig\\
also $y = f(x)$\\
insbesondere $y \neq \infty$ also $f$ beschränkt\\
Für alle $x' \in D$ gilt $f(x') \leq sup \{f(D)\} = y = f(x)$\\
Setze $x_{max} := x$. Dann $f(x') \leq f(x_{max})$ für alle $x' \in D$\\
Anfang findet man $x_{min}$ \qed
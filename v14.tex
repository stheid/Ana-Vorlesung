% Kopfzeile beim Kapitelanfang:
\fancypagestyle{plain}{
%Kopfzeile links bzw. innen
\fancyhead[L]{\calligra\Large Vorlesung Nr. 14}
%Kopfzeile rechts bzw. außen
\fancyhead[R]{\calligra\Large 26.11.2012}
}
%Kopfzeile links bzw. innen
\fancyhead[L]{\calligra\Large Vorlesung Nr. 14}
%Kopfzeile rechts bzw. außen
\fancyhead[R]{\calligra\Large 26.10.2012}
% **************************************************
\wdh{Logarithmus und allgemeine Potenzen}
Die Funktion $exp: \R \to \R_{\geq 0}$ ist stetig, streng monoton wachsend, bijektiv.\\*
Die Umkehrfunktion heißt \ul{Logarithmus},\\*
$$log = exp^{-1}: \R_{\geq 0} \to \R$$
explizit definiert durch $log(x) = y \equ x = exp(y)$\\*
$\underset{\text{\scriptsize{nach satz 6.5}}}{\Rarr}$ log ist stetig, streng monoton wachsend, bijektiv.
% GRAPH exp(x) | GRAPH log(x)
\sS{Satz (Eigenschaften des Logarithmus)}
\begin{enumerate}
\item{$log(1)=0$}
\item{$log(x·y)=log(x)+log(y)$}
\item{$\lim\limits_{x→0}log(x)=-∞$}
\item{$\lim\limits_{x→∞}log(x)=∞$}
\end{enumerate}
\bew
Folgt aus Eigenschaften von $exp$, Details: Übung
\sss{Erinnerung:}
also $a>0,\ n\eZ,\ m\eN$ ist $a^{\frac{n}{m}}:=\sqrt[m]{a^n}$
\sS{Lemma}
Es gibt $a^{\frac{n}{m}} =  exp(\frac{n}{m} \cdot log(a)$\\*
\bew
	\begin{enumerate}
	\item{Sei $n \geq 0$:\\*
	$$exp(n \cdot log(a)) = exp(\overbrace{log(a) + log(a) + ... + log(a)}^{n}) = exp(log(a)) \cdot ... \cdot exp(log(a))$$}
	\item{Sei $n < 0$\\*
	 $$exp(n \cdot log(a)) = exp(\underbrace{-n}_{-n > 0} \cdot log (a)) = (a^{-1})^{-1} = a^n$$}
	 \item{Rechne: $exp(\frac{n}{m} log(a))^m = exp(m \cdot \frac{n}{m} \cdot log(a)) = exp(n \log(a)) = a^n$}
	\end{enumerate}	 
	$\sqrt[m]{\ } \Rarr exp(\frac{n}{m} log(a)) = \sqrt[m]{a^n} = a^{\frac{n}{m}}$
\sS{Definition}
Sei $a>0,\ x\eR$ setze $a^x:=exp(x·log(a))$\\*
\sS{Bemerkung}
Die Regeln der Potenzrechnung gelten:
$$a^{x+y}=a^x·a^y,\qquad a^{x·y}=(a^x)^y$$
\bew
$$a^{x+y}=exp((x+y)·log(a))=exp(x·log(a))·exp(y·log(a))=a^x·a^y$$
$$a^{x·y}=exp(x·y·log(a))=(a^x)^y=exp(y·log(exp(x·log(a))))\footnote{$log\circ exp= id$}=exp(y·x·log(a))$$\qed
\bem
Eulersche Zahl
$$e:=exp(1)=2{,}7…\footnote{$log(e)=1$}\qquad e^x=exp(x·log(e))=exp(x)$$

\sS{Definition Logarithmusbasis}
Sei $a > 1 \qquad x \in \R$ 
$$log_a (x) = \frac{log(x)}{log(a)}$$
\bem $a \neq 0 \Rarr log(a) \neq 0$\\*
Dann $a^{log_a (x)} = exp(log_a (x) \cdot log(a)) = exp(\frac{log(x)}{log(a)} \cdot log(a)) = exp(log(x)) = x$
\uS{Gleichmäßige Stetigkeit}
\wdh
$f: D \to \R$ stetig an $x \in D$ wenn gilt:\\*
Für jedes $\e > 0$ gilt $\delta > 0$ mit wenn $y \in D$ mit $|x - y| < \delta$ dann $|f(x) - f(y)| < \e$\\*
Hier hängt $\delta$ im allgemeinen von $\e$ und $x$ ab!
%
\sS{Definition:}
Eine Funktion $f:D→\R$ heißt gleichmäßig stetig wenn gilt:\\*
Für jedes $ε>0$ gibt es ein $δ>0$ so dass für alle $x,y\in D$ mit $|x-y|<δ$ gilt:
$$|f(x)-f(y)|<ε$$
\sss{Wesentlich:}
$δ$ hängt nur von $ε$, nicht von $x$ ab.
\bsp
$D=(0,1)\qquad f:D→\R,\ f(x)=\frac{1}{x}$\\*
GRAPH $f$ stetig, aber nicht gleichmäßig stetig
\bew
Wähle $ε=1$. Angenommen es gibt $δ>0$ mit $|x-y|<δ\Rarr |f(x)-f(y)<1$
Wähle: $x=\frac{1}{n},\ y\frac{1}{n+1}$ so dass $\frac{1}{n·(n+1)}<δ$
$$|x-y|=|\frac{1}{n}-\frac{1}{n+1}|=|\frac{n+1-n}{n·(n+1)}|=\frac{1}{n·(n+1)}$$
Dann $$|f(x)-f(y)|=|n-(n+1)|=1$$
Das Zeigt: $δ$ existiert nicht.

\sS{Satz}
Seien $a \leq b$ reelle Zahlen\\*
Jede stetige Funktion $f: [a, b] \to \R$ ist gleichmäßig stetig.\\*
\bew
	Angenommen, $f$ ist nicht gleichmäßig stetig:
	Es gibt ein $\e > 0$ sodass für jedes $\delta > 0$ zwei Zahlen $x, y \in D$ existieren, mit $|x - y| < \delta$ und $|f(x) - f(y)| \geq \e$\\*
	Wähle zu $\delta = \frac{1}{n}$ Zahlen $x_n, y_n \in D$ mit $|x_n - y_n| \frac{1}{n}, \ |f(x) - f(y)| \geq \e$ (*)\\*
	Bolzano-Weierstraß \Rarr Es gibt eine Teilfolge $({x_n}_k)_{k\geq 0}$, die konvergiert. Sei $x = \ds\lim_{k\to \infty}({x_n}_k)_{k\geq 0} \in  [a, b] = D$\\*
	Dann $\ds\lim_{k \to \infty} ({y_n}_k)_{k\geq 0} = \ds\lim_{k \to \infty} (({y_n}_k)_{k\geq 0} - ({x_n}_k)_{k\geq 0}) = 0 + x = x$\\*
	${x_n}_k)_{k\geq 0} \to x, \ {y_n}_k)_{k\geq 0} \to x \ f: k \to \infty $\\*
	$f$ stetig \Rarr $f({x_n}_k)_{k} \to f(x), f({y_n}_k)_{k} \to f(x) \ f \ k \to \infty $\\*
	\Rarr $(f({x_n}_k)_{k}) - f({y_n}_k)_{k})  \to f(x) - f(x) = 0$
	Widerspruch zu (*) Also ist $f$ gleichmäßig stetig. \qed


\chapter{Komplexe Zahlen und Trigonometrie}
Der Körper \C{} der komplexen Zahlen\\*
Mangel von \R{}: Die Gleichung $x^2=-1$ hat keine Lösung\\*
\sS{Definition}
Es sei $\C=\R^2=\{(x,y)|x,y\eR\}$ mit folgender Addition und Multiplikation:
$$(x,y)+(x',y'):=(x+x',y+y')$$
$$(x,y)·(x',y'):=(x·x'-y·y',x·y'-y·x')$$
Addition gleich der Vektoraddition in \R^2 GRAPH
\sS{Satz}
\C{} ist ein Körper mit Null (0,0) und Eins (1,0)
\bew
Überprüfe Körperaxiome
\ssss{Exemplarisch:}
\begin{enumerate}
\item{Assotiativgesetz der Multiplikation\\*
Gegeben $(x,y),\ (x',y'),\ (x'',y'')\eC$
$$((x,y)·(x',y'))·(x'',y'')=(x·x'-y·y',x·y'-x·y'+y·x')(x'',y'')=(x·x'·x''-y·y'·x''-x·y'·y''-y·x'·y'',x·x'·y''-y·y'·y''+x·y'·x''+y·x'·x'')$$
$$(x,y)((x',y')(x'',y'')=…\text{ erhalte gleiches Ergebnis}$$}
\item{Existenz vom Inversen:\\*
Sei $z = (x, y) \in \C, \ x, y \in \R, \ z \neq 0$ \\*
Zeige es gibt ein $z^{-1} \in \C$ mit $z \cdot z^{-1} = (1,\ 0)$\\*
$z \neq 0 \Rarr x \neq 0$ oder $y \neq 0 \equ x^2 + y^2 > 0$\\
$$w:= \left(\frac{x}{x^2 + y^2}, \frac{-y}{x^2 + y^2}\right)$$
Rechne $z \cdot w = (x, y) \cdot \left(\frac{x}{x^2 + y^2}, \frac{-y}{x^2 + y^2}\right)$
$=\left(\frac{x^2}{x^2 + y^2} - \frac{-y^2}{x^2 + y^2}, \frac{-yx}{x^2 + y^2} + \frac{yx}{x^2 + y^2}\right) = (1, 0)$
Also $w = z^{-1}$ \qed \\*
Weitere Axiome ähnlich.}
\end{enumerate}
\sss{Definition:}
$i:=(0,1)$ (imaginäre Einheit)\\*
Dann ist $$i^2=(0,1)·(0,1)=(-1,0) \Rarr i^2+1=0$$
\bem
Für $x,x'\eR$ gilt:
$$(x,0)+(x',0)=(x+x',0)$$
$$(x,0)·(x',0)=(x·x',0)$$
Die Abbildung $\R→\C,\ x\mapsto(x,0)$ ist injektiv und verträglich mit $+,·$\\*
$\leadsto$ Fasse \R{} mittels diese Abbildung als Teilmenge von \C{} auf, einschließlich der Körperstruktur\\
Dann $i^2=-1$.\\*
Für
$(x,y)\eC$ gilt $(x,y)=(x,0)+(0,y)=(x,0)+(0,1)·(y,0)=x+i·y$
Jede komplexe Zahl $z$ hat eine eindeutige Darstellung $z=x+i·y$ mit $x,y\eR$.\\
\sss{Idee}
\C{} entsteht aus \R{} durch Hinzunahme einer Zahl $i$ mit $i^2=-1$\\
Interpretation der Multiplikation in \C:\\*
$$(x+i·y)(x'+i·y')=(x·x'+x·i·y'+i·y·x'+i·y·i·y')=(x·x'-y·y')+i(x·y'+y·x')$$
\sss{Definition}
Sei $z = x + iy \in \C$\\*
Realteil: Re(z) := x \\*
Imagnärteil: Im(z) := y\\*
Komplex konjugierte Zahl $\={z} = x - iy$\\*
Komplex konjugation = Spiegelung an der x-Achse.\\*
\ul{Definition:} Der Betrag von $z = x + iy \in \C$ ist $|z| = \sqrt{x^2 + y^2}$ Abstand von $0 = (0, 0)$ zu $z$\\*
\bem
\begin{enumerate}
\item{$$z·\={z}=(x+i·y)(x-i·y)=x^2+y^2+i(-x·y+y·x)=x^2+y^2=|z|^2$$}
\item{Insbesondere gilt $z·\={z}\eR$ und $z·\={z}\geq 0$}
\item{$|z|=\sqrt{x^2+y^2=}=\sqrt{z·\={z}}$ (sinnvoll wegen 2)}
\end{enumerate}
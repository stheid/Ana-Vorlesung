% Kopfzeile beim Kapitelanfang:
\fancypagestyle{plain}{
%Kopfzeile links bzw. innen
\fancyhead[L]{\calligra\Large Vorlesung Nr. 26}
%Kopfzeile rechts bzw. außen
\fancyhead[R]{\calligra\Large 24.01.2013}
}
%Kopfzeile links bzw. innen
\fancyhead[L]{\calligra\Large Vorlesung Nr. 26}
%Kopfzeile rechts bzw. außen
\fancyhead[R]{\calligra\Large 24.01.2013}
% **************************************************
%
\wdh
\usS{Definition Potenzreihe (10.1)}
$P(z)=\Sum_{n=0}^{∞}a_nz^n,\ a_n\eC,$ falls $a_n\eR$ reelle Potenzreihe\\*
%Vielleicht F(x)
$F(x): exp(z)=\Sum_{n\geq 0}\frac{1}{n!}z^n,\ cos(x)=\Sum_{n\geq 0}(-1)^n\frac{1}{(2n)!}x^{z^n},\ sin(x)=…$
%christopher
\bsp
\enum{
\item $\Sum_{n\geq 0}2^nz^n=\Sum_{n\geq 0}(2z)^n$ (geometrische Reihe, konvergiert \equ\ $|2z|<1$ also $|z|<\frac{1}{2}$)\\*
Konvergenzradius: $a_n=2^n\ \Rarr\ \sqrt[n]{|a_n|}=2\ \Rarr\ \limsup\sqrt[n]{|a_n|}=2\ \Rarr$ Konvergenzradius=$\frac{1}{2}$
\item $exp(z)=\Sum_{n\geq 0}\frac{z^n}{n!},\ a_n\frac{1}{n!}$ Konvergenz für alle $z$\\
$\limsup(\sqrt[n]{\frac{1}{n!}}=0$\\
Zeige $\sqrt[n]{n!}\underset{n\to\infty}{\longrightarrow}∞$\\*
z.B. $(n!)^2=1·2·…·n·1·2·…=(1·n)(2·(n-1))(3·(n-2))·(n·1)\geq n^n$
\item $cos(x)=\sum\frac{(1)^k}{(2-n)!}x^{2n}$ konvergent für alle $x$ \Rarr\ Konvergenzradius=∞
%christopher 11 taylorreihen
% definition
\sS{Satz (Lagrangeform von $R_{n+1}$}
Sei $f$ wie oben und $x\in I$. Es gibt in $x_0$ zwischen $a$ und $x$, so dass
$$R_{n+1}(x)=\frac{f^{(n+1)}(x_0)}{(n+1)!}(x-a)^{n+1}$$
Skizze auf der rechten tafelseite 5
\sss{Achtung} $x_0$ hängt von $x$ ab!
\bsp
$f(x)=cos(x),\ a=0$\\*
\Rarr $$|cos(x)-T_n(x)|=|R_{n+1}(x)|\underset{Lagrange}{=}\left|\frac{f^{(n+1)}(x_0)}{(n+1)!}(x-a)^{n+1}\right|\underset{\overset{|cosx_0|<0}{|sinx_0|<0}}{\leq}\frac{|x|^{n+1}]}{(n+1)!}\underset{\nif}{\longrightarrow}0$$
dass heißt $cos(x)$ kann beliebig gut durch $T_n$ approximiert werden.

\sS{Definition Tailorreihen}
Die Tailorreihe von $f$ in $a$ ist
$$T(x):=T_{f,a}(x):=\sum_{n\geq 0}\frac{f^{(n)}(a)}{n!}(x-a)^n$$
(ist Potenzreihe mit zugehörigem Konvergenuzradius)
\bsp
\enum{
\item $f(x)=\exp(x),\ f^{(k)}(x)=\exp(x)$\\*
\ary{\Rarr\ &T_{exp,0}(x)=\sum\frac{1}{n!}x^n(=exp(x))\\&T_{exp,0}(x)=\sum\frac{e^a}{n!}(x-a)^n}
\item $f(x)=\sqrt{x},\ a=1\quad f'(x)=\frac{1}{2\sqrt{x}},\ f''(x)=\frac{1}{2}·(-\frac{1}{2})x^{-\frac{3}{2}}_{T_0}$\\*
%t... Christopher
$f^{(3)}(x)=\frac{(-1)^2}{2^3}1·3·x^{-\frac{5}{2}},\ f^{(4)}(x)=\frac{(-1)^3}{2^4}1·3·5·x^{-\frac{7}{2}},\  f^{(n)}(x)=\frac{(-1)^{n-1}}{2^n}1·3·5·…·(2n-3)·x^{-\frac{2n-1}{2}}$
\Rarr $T_{\sqrt,1}(x)=\sum(-1)^{n-1}\frac{1·3·5·7·…·(2n-3)}{2·4·6·…·(2n)}(x-1)^n$ (Konvergenzradius=1)
}
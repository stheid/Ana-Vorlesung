\documentclass[a4paper,10pt]{scrreprt}
\usepackage[utf8]{inputenc}
\usepackage{amsmath,amssymb}
\usepackage{textcomp}
\usepackage[T1]{fontenc}
\usepackage{breqn}
\usepackage{dsfont}
\usepackage{calligra}
\usepackage{cancel}
\usepackage{tipa}

%Beschreibaren Seitenbereich definieren
\usepackage[left=2.5cm,right=2.5cm,top=2cm,bottom=2cm]{geometry}

%Kopf- und Fußzeile einfügen
\usepackage{fancyhdr}
\pagestyle{fancy}

%Kopfzeile Linie oben
\renewcommand{\headrulewidth}{0.5pt}

%Fußzeile mittig
\fancyfoot[C]{\thepage}

%Linie unten
\renewcommand{\footrulewidth}{0.5pt}

% *************************
% *** costum characters ***
% *************************

% Unicode Character definition
\DeclareUnicodeCharacter{221E}{\text{$\infty$}}
\DeclareUnicodeCharacter{00B7}{\text{$\cdot$}}
\DeclareUnicodeCharacter{2192}{\text{$\to$}}
\DeclareUnicodeCharacter{2026}{\text{\ldots}}
\DeclareUnicodeCharacter{00D7}{\text{$\times$}}
\DeclareUnicodeCharacter{2254}{\text{:=}}

% ***********************
% *** costum commands ***
% ***********************

% kürzerer command um neue mathemathische commands (ohne parameter) zu erstellen
\newcommand{\mcmd}[2]{\newcommand{ #1 }{\text{$#2$}}} 
%kürzerer command um neue text commands (ohne parameter) zu erstellen
\newcommand{\tcmd}[2]{\newcommand{ #1 }{\text{#2}}} 
%kürzerer command um neue commands (ohne parameter, ohne environment) zu erstellen
\newcommand{\cmd}{\newcommand}


% Mengen
\mcmd{\N}{\mathds{N}}
\mcmd{\Z}{\mathds{Z}}
\mcmd{\Q}{\mathds{Q}}
\mcmd{\R}{\mathds{R}}
\mcmd{\C}{\mathds{C}}

% griechische Buchstaben
\mcmd{\z}{\zeta}
\mcmd{\e}{\varepsilon}	

% Pfeile
\mcmd{\equ}{\Leftrightarrow}
\mcmd{\Rarr}{\Rightarrow}
\mcmd{\Larr}{\Leftarrow}

% Symbole
\cmd{\qed}{\hfill\text{$\blacksquare$}}
\mcmd{\bs}{\backslash}
\mcmd{\ba}{\backslash}

% Formelkürzel
\cmd{\bino}[2]{\text{$\ds\binom{#1}{#2} = \frac{#1!}{#2!\cdot(#1-#2)!}$}}

% Überschriften
\cmd{\sS}[1]{\section{#1}}
\cmd{\Def}{\sS{Definition:}}
\cmd{\Satz}{\sS{Satz:}}
\cmd{\uS}[1]{\section*{\underline{#1}}}
\cmd{\wdh}{\section*{Wiederholung}}
\cmd{\sss}[1]{\vspace{-4mm}\subsection*{\underline{#1}}}
\cmd{\bsp}{\sss{Beispiel:}}
\cmd{\Bsp}[1]{\vspace{-4mm}\subsection*{\underline{Beispiel:} #1}}
\cmd{\bem}{\sss{Bemerkung:}}
\cmd{\bew}{\sss{Beweis:}}
\cmd{\Bew}[1]{\vspace{-4mm}\subsection*{\underline{Beweis} #1:}}
\cmd{\anm}{\sss{Anmerkung:}}
\cmd{\ssss}[1]{\vspace{4mm}{\bf\underline{#1}}\\}

% Textkürzel
\cmd{\ok}{\marginpar{\checkmark}}
\cmd{\einruck}[2]{#1\vspace{-12pt}\begin{addmargin}{.05\textwidth}#2\end{addmargin}}
\cmd{\ind}[3]{\einruck{IA:}{#1\ok}\\\einruck{IV:}{#2}\\\einruck{IS: }{$n\to n+1$#3\qed}}
\cmd{\notat}[1]{\einruck{\em Notation:}{\em#1}}

% Environments
\cmd{\ds}{\displaystyle}


% Author, title…
\title{Analysis Vorlesung}
\author{Stefan Heid, Christopher Jordan}
\date{\today}

% Less detailed TOC
\setcounter{tocdepth}{1}

\renewcommand*{\contentsname}{Inhaltsverzeichnis}

\begin{document}
%\maketitle
%\tableofcontents
%\newpage
%% Kopfzeile beim Kapitelanfang:
\fancypagestyle{plain}{
%Kopfzeile links bzw. innen
\fancyhead[L]{\calligra\Large Vorlesung Nr. 1}
%Kopfzeile rechts bzw. außen
\fancyhead[R]{\calligra\Large 8.10.2012}
}
%Kopfzeile links bzw. innen
\fancyhead[L]{\calligra\Large Vorlesung Nr. 1}
%Kopfzeile rechts bzw. außen
\fancyhead[R]{\calligra\Large 8.10.2012}
% **************************************************
\chapter{Mengen}
\Def
\begin{enumerate}
\item Eine Menge ist eine Ansammlung verschiedener Objekte
\item Die Objekte in einer Menge heißen \underline{Elemente}\\
%
\notat{
a $\in$ M heißt a ist Element der Menge M\\
a ${\not\in}$ M heißt a ist kein Element der Menge M}
%
\item Sei M eine Menge. Eine Menge U heißt Teilmenge von M, von der jedes Element von U auch Element von M ist\\
%
\notat{
U $\subseteq$ M heißt U ist Teilmenge von M\\
U ${\not\subseteq}$ M heißt U ist keine Teilmenge von M}
\end{enumerate}
%
\sS{Beispiele}
\begin{enumerate}
\item {\einruck{Sei}{M die Menge aller Studierenden in L1\\W  die Menge aller weiblichen Studierenden in L1\\F die Menge aller Frauen}
Dann gilt: W $\subseteq$ M, W $\subseteq$ F, M ${\not\subseteq}$ F, F ${\not\subseteq}$ M}
\item {Die Menge der natürlichen Zahlen
$\N = \{1,2,3,4 …\}$
G sei die Menge der geraden natürlichen Zahlen
$G := \{n \in \N | $n ist gerade$\} = \{2m | m \in \mathds{N}\} = \{2,4,6,8 …\}$
Es gilt G $\subseteq \N, \N \subseteq$ G}
\item {Die Menge der ganzen Zahlen
$\Z = \{0,1,-1,2,-2,3,-3, …\}$}
\item {Die Menge der rationalen Zahlen
$\mathds{Q} = \{a/b | a, b \in \mathds{Z}, b \neq 0\}$}
\item {Die Menge ohne Element heißt die leere Menge
Symbol: $\emptyset = \{\}$}
\end{enumerate}
%
\bem
\begin{enumerate}
\item Für jede Menge M gilt $\setminus \subseteq M$
\item $\N \subseteq \Z \subseteq \Q$
\end{enumerate}

\sS{Definition: Sei M eine Menge und U,V $\subseteq$ M Teilmengen}
\begin{enumerate}
\item Die Vereinbarung von U und V ist $U \cup V := \{x \in M \mid x \in U oder x \in V\}$
\item Der Durchschnitt von U und V ist $U \cap V := \{x \in M \mid x \in U oder x \in V\}$
U und V heißen disjunkt, wenn $U \cap V = \emptyset$
\item Die Differenzmenge von U und V ist $U \setminus V := \{x \in U \mid x \in V\}$
\item Das Komplement von U ist $U^C = M \setminus U = \{x \in M \mid x {\not\in} U\}$
%
\einruck{Bsp: }{Sei M = N \\
$\{1,3\} \cup \{3,5\} = \{1,3,5\}$\\
$\{1,3\} \cap \{3,5\} = \{3\}$\\
$\{1,3\} \cap \{2,4,7\} = \emptyset \leftarrow$ disjunkt\\
$\{1,2,3\} \setminus \{3,4,5\} = \{1,2\}$\\
$\{1,3,5\}^C = \{2,4,6,7,8,…\}$}
\end{enumerate}
%
\sS{Satz (de Morgan'sche Regeln)}
Sei M eine Menge, U,V $\subseteq$ M Teilmengen\\
Dann:
\begin{enumerate}
\item $(U \cap V)^C = U^C \cup V^C$
\item $(U \cup V)^C = U^C \cap V^C$
\end{enumerate}
%
\bew
\begin{enumerate}
\item{Sei $x \in M$\\Es gilt: $x \in (U \cap V)^C \Leftrightarrow x {\not\in} U \cap V \Leftrightarrow x {\not\in} U$ oder $x {\not\in} V\Leftrightarrow x \in U^C$ oder $x \in V^C \Leftrightarrow x\in U^C \cup V^C$}
\item{ Sei $x \in M$\\ Es gilt: $x \in (U \cup V)^C \Leftrightarrow x {\not\in} U \cup V \Leftrightarrow x {\not\in} U$ und $x {\not\in} V\Leftrightarrow x \in U^C$ und $x \in V^C \Leftrightarrow x\in U^C \cap V^C$}
\end{enumerate}
%
\section{Prinzip der Vollständigen Induktion}
Für jedes $n \in \N$ sei eine Aussage A(n) gegeben\\
Ziel: Beweisen, Dass A(n) für jedes $n \in \N$ mehr ist dafür reicht es zu zeigen
\begin{enumerate}
\item{Induktionsanfang (IA): A(1) ist wahr}
\item{Induktionsschrit (IS): Wenn für ein $n \in \N$ A(n) wahr ist, dann ist auch A(n+1) wahr}
\end{enumerate}
%
\Satz
Für jede natürliche Zahl n gilt: $\ds 1+2+3+4+5+…+n=\frac{n(n+1)}{2}$\\
Probe:\\
\begin{tabular}{r|c|c|c|c}
n & 1 & 2 & 3 & 4\\ \hline\hline
1+2+3...+n & 1 & 3 & 6 & 10\\ \hline
$\ds \frac{n(n+1)}{2}$ & 1 & 3 & 6 & 10\\
\end{tabular}
\sss{Beweis des Satzes mit Induktion}
Abkürzung: $S(n) := 1+2+3+…+n$
Aussage: A(n): $\ds S(n) = \frac{n(n+1)}{2}$
\begin{enumerate}
\item {Induktionsanfang (IA): n=1 $S(1) = 1 = \dfrac{1·2}{2}$\marginpar{ok!}}
\item {Induktionsschritt (IS): $n → n+1$\\
Annahme: A(n) gilt: $\ds S(n) = \frac{n(n+1)}{2}$\\
Zu zeigen: A(n+1) gilt: $S(n+1)=\frac{(n+1)\cdot(n+2)}{2}$\\
$\ds S(n+1)=S(n)+n+1=\frac{n(n+1)}{2}+\frac{2(n+1)}{2}=\frac{(n+2)(n+1)}{2}$\\
Das beendet den Beweis} \qed
\end{enumerate}
Zur Vereinfachung der Notation:\\
Seien $a_1,a_2,a_3,...,a_n$ Zahlen $n \in \N$\\
Setze: $\sum_{k=1}^n a_k := a_1+a_2+a_3+…+a_n$\\
\begin{tabbing}
Allgemeiner: \=Sei $l,m \in \mathds{N}$, $l \le m \le n$\\
\>$\sum_{k=l}^m a_k = a_l+a_{l+1}+…+a_m$\\
\end{tabbing}
Aussage des Satzes:
\[ \sum_{k=1}^n k = \frac{n(n+1)}{2}\]
\hfill\underline{Kombinatorik} (mathematisches Zählen)

\Def
Seien A, B Mengen. Das kartesische Produkt von A und B ist definiert als $A × B := \{(a,b)|a\in A, b \in B\}$ Die Elemente von $A × B$ heißen geordnete Paare\\
Bsp.: $\{1,7\}\times \{2,3\}=\{(1,2),(1,3),(7,2),(7,3)\}$\\
Allgemeiner: Gegeben seien Mengen
$A_1,…,A_k$ mit $k \in \N$. Das kartesische Produkt von $A_1,…,A_k$ ist $A_1\times …\times A_k = \{(a_1,…,a_k)|a\in A, $für $i=1,…,k\}$\\
Elemente von $A_1 × … × A_k$ heißen k-Tupel\\
Falls $A_1=A_2=…=A_k=A$, schreibe $\underbrace{A\times…\times A}_{k-mal}=A^k$

\section{Definition}
Eine Menge A ist endlich, wenn A nur endlich viele Elemente hat. Dann bezeichnet
$\#A = \{|A|\}$ die Anzahl der Elemente von A und somit dessen Kardinalit\"at
oder M\"achtigkeit. Wenn A nicht endlich ist, so schreibe: $\# A= \infty$\\
Bsp.: $\#\emptyset = 0, \#\mathds{N}=\infty, \# \{1,3,5\} = 3$

\section{Bemerkung}
\begin{enumerate}
\item Sei A endliche Menge. $U,V\subseteq A$ disjunkte Teilmengen\\
Dann $\#(U\cup V)=\# U + \# V$ 
\item Seien $A_1,...,A_k$ endliche Mengen $k \in \mathds{N}$\\
Dann: $\#(A_1 \times ... \times A_k)=(\#A_1)(\#A_2)...(\#A_k)$
\end{enumerate}

\section{Definition}
\begin{enumerate}
\item Für $n\in \mathds{N}$ setze $n!=1\cdot 2\cdot 3\cdot ... \cdot n=\prod_{k=i}^n k$
Setze $0!=1$
\item Für $k,n\in \Z$ mit $0\le k \le n$ sei ${n \choose k} := \frac{n!}{k!\cdot(n-1)!}$ $\leftarrow$ Binomialkoeffizient\\
\begin{tabular}{r|c|c|c|c|c|c|c}
n & 0 & 1 & 2 & 3 & 4 & 5 & 6\\ \hline
n! & 1 & 1 & 2 & 6 & 24 & 120 & 720
\end{tabular}\\
\bsp
${5 \choose 2} := \frac{5!}{2!\cdot 3!} = \frac{5\cdot 4 \cdot\cancel{3 \cdot 2 \cdot 1}}{2\cdot 1\cdot \cancel{3\cdot 2}\cdot 1 } = \frac{20}{2}=10$\\
Bemerkung: ${ n \choose 0 }= 1 = {n \choose n}$
\end{enumerate}\newpage
%\input{v2}\newpage
%\chapter{Angeordneter Körper}
%\chapter{Folgen}
%\chapter{Konvergenzsätze}
%%Kopfzeile links bzw. innen
\fancyhead[L]{\calligra {\Large Vorlesung Nr. 6}}
%Kopfzeile rechts bzw. außen
\fancyhead[R]{\calligra 25.10.2012}
% *****************************************
%
\setcounter{chapter}{3}
\setcounter{section}{9}
%
\wdh
Eine Folge reeller Zahlen $(a_n)$ konvergiert uneigentlich gegen ∞ wenn gilt:\\
Für jedes $C \in \R$ gibt es ein $n \in\N$ mit $a_n > C$ für jedes $n \in\N$\\
\\
$(a_n)$ konvergiert uneigentlich gegen $- \infty$ wenn $(-a_n)$ gegen $\infty$ konvergiert.\\
%
\notat{
$a_n \to \infty \qquad \text{ für } n \to \infty$\\
$a_n \to - \infty \qquad \text{ für } n \to \infty$
}
%
\bsp
$a_n = n^2 \to \infty$\\
$a_n = -n^2 \to -\infty$\\
$a_n = (-1)^n \cdot n^2$\\
$(0, -1, 4, -9)$ konvergiert weder gegen $\infty$ noch gegen $ - \infty$
%
\sss{Rechenregeln:}
Angenommen $(a_n), (b_n)$ sind konvergente Folgen.\\
\begin{enumerate}
\item{$(a_n + b_n) \to a + b$}
\item{$(a_n \cdot b_n) \to ab$}
\item{$\ds\frac{1}{b_n} \to \frac{1}{b}$}
\item{$c \cdot a_n \to c \cdot a$}
\item{$a_n - b_n \to a - b$}
\item{$\ds\frac{a_n}{b_n} \to \frac{a}{b}$}
\end{enumerate}
%
\Bew{6}
3) $\Rightarrow \displaystyle\frac{1}{b_n} \to \displaystyle\frac{1}{b}$\\
$\displaystyle\frac{a_n}{b_n} = a_n \cdot \displaystyle\frac{1}{b}$\\
2) $\Rightarrow a_n \cdot displaystyle\frac{1}{b_n} \to a \cdot \displaystyle\frac{1}{b} = \displaystyle\frac{a}{b} \phantom{XXX} q.e.d.$\\
%
\bsp
\begin{tabular}{l|c|c|c|c|c|r}
$n$   & 0 & 1 & 2 & 3 & 10 & 100\\\hline
$a_n$ & 0 & 0 & $\frac{2}{9}$ & $\frac{6}{19}$ & $\frac{90}{201}$ & $\frac{9900}{20001}$ \\
\end{tabular}
\vspace{5mm}
Vermutung: $a_n \to \displaystyle\frac{1}{2}$ für $n \to \infty$\\
Rechenregel 6 anwenden:\\
\begin{itemize}
\item[1.]{Versuch:\\
$a_n = \frac{b_n}{c_n}$\\ 
\\
$b_n = n^2 -n; c_n = 2n^2 + 1$\\
$(b_n) und (c_n)$ sind divergend. Schlecht.}
\item[2.]{Versuch:\\
$\displaystyle\frac{n^2 - n}{2n^2 + 1} = \displaystyle\frac{n^2(1 - \displaystyle\frac{1}{n})}{n^2(2 + \displaystyle\frac{1}{n^2}}$ für $n \geq 1$ \\ \\
$= \displaystyle\frac{1 - \frac{1}{n}}{2 + \frac{1}{n^2}} = \displaystyle\frac{b_n}{c_n}$ mit $b_n := 1 - \displaystyle\frac{1}{n}; c_n = 2 + \displaystyle\frac{1}{n^2}$\\
\\
$\displaystyle\frac{1}{n} \to 0$ für $n \to \infty $
\\ \\ $\Rightarrow 1 - \displaystyle\frac{1}{n} \to 1 - 0 = 1$ für $n \to \infty$
\\ \\ $\Rightarrow 2 + \displaystyle\frac{1}{n^2} \to 2 + 0 = 2$ für $n \to \infty$}
\end{itemize}

$\Rightarrow a_n \to \frac{1}{2}$ für $n \to \infty$\\

\sS{Satz}
Seien $a_n \to a$, $b_n \to b$ zwei konvergente Folgen reeler Zahlen.\\
wenn $a_n \leq b_n$ für unendlich viele $n \in \mathbb{N}$ dann ist $a \leq b$. \\
\bew
Angenommen: $a > b$\\

Wähle $\epsilon := \displaystyle\frac{a - b}{2} > 0$\\
Es gibt $N \in \mathbb{N}$ so dass:
$
\left.
\begin{array}{ll}
\mid a_n - a \mid  < \epsilon \\
\mid b_n - b \mid  < \epsilon
\end{array} \right\rbrace$ für $n \geq N$\\
$\Rightarrow a_n > a - \epsilon$\\ \\
$= a - \frac{a - b}{2} = \displaystyle\frac{a + b}{2} = b + \displaystyle\frac{a - b}{2}\\
\\
= b + \epsilon > b_n \Rightarrow a_n > b_n$ für $n \geq \mathbb{N}$\\
Widerspruch zur Annahme.\\
$a_n \leq b_n$ für unendlich viele $n \in \mathbb{N} \hfill q.e.d.$

\sS{Definition: Reihen}
Sei $(a_n)_{n \geq 0}$ eine Folge reeler Zahlen.\\
Bilde eine Folge:
\begin{align*}
s_0 &= a_0\\
s_1 &= a_0 + a_1\\
s_2 &= a_0 + a_1 + a_2\\
\vdots
s_n &= a_0 + a_1 + a_n = \sum\limits_{k = 0}^{n} a_k
\end{align*}
Die Folge $(s_n)_{n \geq 0}$ heißt Reihe mit den Gliedern $a_n$.\\
$s_n$ heißen die \underline{Partialsummen} der Reihe.\\
Bezeichnung:\\
$\sum\limits_{k = 0}^{\infty} a_k$ oder $a_0 + a_1 + a_2 + a_3 + ...$\\ \\
Wenn $s_n \to s \in \mathbb{R}$ für $n \to \infty$ dann schreiben wir:\\
$\sum\limits_{k = 0}^{\infty} a_k = s$\\
Summe der Reihe.\\
\\
\underline{Achtung:} Symbol $\sum\limits_{k = 0}^{\infty} a_k$ hat \underline{zwei} Bedeutungen:
\begin{enumerate}
\item{die Folge $(s_n)$} \\
oder 
\item{deren Grenzwert}
\end{enumerate}
\bsp
\begin{enumerate}
\item{$\sum\limits_{k = 1}^{\infty} 1 = 1+1+1+...$\\
ist die Folge $(1, 2, 3, 4,...) = (n + 1)_{n \in \mathbb{N}_{0}}$}
\item{$\sum\limits_{k = 1}^{\infty} k = 0 + 1 + 2 + 3+ ...$ \\
ist die Folge $(1, 3, 6, 10,...) = (\displaystyle\frac{n(n - 1)}{2})_{n \in \mathbb{N}}$ }
\item{$\sum\limits_{k = 1}^{\infty} \displaystyle\frac{1}{k(k+1)} = \displaystyle\frac{1}{2} + \displaystyle\frac{1}{6} + \displaystyle\frac{1}{12} + ...$\\
ist die Folge $(\displaystyle\frac{1}{2}, \displaystyle\frac{2}{3}, \displaystyle\frac{3}{4})$}
\end{enumerate}
\vspace{5mm}
Vorüberlegung:\\
$\displaystyle\frac{1}{k(k+1)} = \displaystyle\frac{(k+1) - k}{k(k+1)} = \frac{1}{k} - \displaystyle\frac{1}{k + 1}$\\ \\
$s_n := \sum\limits_{k = 1}^{\infty} \displaystyle\frac{1}{k(k+1)}
= (\displaystyle\frac{1}{1} - \displaystyle\frac{1}{2}) + (\displaystyle\frac{1}{2} - \displaystyle\frac{1}{3}) + ... + (\displaystyle\frac{1}{n} - \displaystyle\frac{1}{n + 1})\\ \\
= 1 - \displaystyle\frac{1}{n + 1}\\ $ Teleskopsumme \\ \\
$\displaystyle\frac{1}{n + 1} \to 0$ für $n \to \infty$\\ \\
Summe der Reihe:\\ \\
$\sum\limits_{k = 1}^{\infty} \displaystyle\frac{1}{k(k+1)} = \lim_{n \to \infty}(1 - \displaystyle\frac{1}{n + 1}) = 1 \phantom{XXX} q.e.d.$\\ 
\\
\bem Jede Folge kann man auch als Reihe Schreiben. (Differenzen bilden)\\
z.B.: die Folge der Primzahlen:\\
$(2, 3, 5, 7, 11, 13, 17, 19)$\\
ist die Reihe:\\
$(2 + 1 + 2+ 4+2+4+2+...)$\\
Goldbachsche Vermutung: in dieser Reihe kommt die Zahl 2 unendlich oft vor.\\
\sS{Satz, Die geometrische Reihe}
Sei $x \in \mathbb{R}$\\
a) $ \sum\limits_{k = 0}^{\infty} x^k = 1 + x^1 + x^2 + x^3 + ... = \frac{1}{1-x} \text{ wenn } \mid x \mid < 1$\\
b) $ \sum\limits_{k = 0}^{\infty} x^k \text{ divergiert wenn } \mid x \mid \geq 1$\\
\begin{itemize}
\item[a] {wenn $|x| < 1$\\
dann folgt $\sum{k=0}{\infty} a_k = \displaystyle\lim_{n \to \infty}(\frac{1}{1 - x} - \frac{x}{1-x} \cdot x^n) = \frac{1}{1 - x}$}
\item[b]{wenn $|x| > 1$\\
dann $(x^n)$ divergent $\Rightarrow (\frac{x}{1-x} \cdot x^n)$ divergent\\
denn $\frac{x}{1-x} \neq 0$\\
$\Rightarrow (\frac{?}{?})$}
\end{itemize}
\bew
$x = 1 \phantom{xxx} \sum_{k = 0}^{\infty} x^k = (1 + 1 + 1 +...)\text{ divergiert, ok}\\
\text{Sei nun }x \neq\\
\text{Bekannt aus der Übung: } \displaystyle\sum_{k = 0}^{\infty} x^k = 1 + x + x^2 +x^3 ... +x^n = \displaystyle\frac{1 -x^{n+1}}{1 - x} = \displaystyle\frac{1}{1 - x} - \displaystyle\frac{x}{1 - x} \cdot x^n \\ $
Potenzenwachstum\\
$x^n \to 0$ für $ n \to \infty$ \underline{wenn} $|x| < 1$\\
$(x^n)$ divergiert, wenn $(|x| \geq 1 \text{ und } x \neq 1)$\\
%
\sS{Satz}
Wenn die Reihe $\displaystyle\sum\limits_{k=0}^{\infty} a_k $ kovergiert, dann ist $(a_n)_{n \in \mathbb{N}}$ eine Nullfolge.\\
\\
\bew Gegeben sei $\epsilon > 0$\\
Sei $a = \displaystyle\sum\limits_{k = 0}^{\infty} a_k = $ $\displaystyle\lim_{n \to \infty}(s_n)$ mit $s_n = a_0 + ... + a_n$\\
Es gibt $ N \ in \mathbb{N}$ mit $|s_n - a| < \displaystyle\frac{\epsilon}{2}$ für $n \geq N$\\
$|a_n| = |s_n - s_{n-1}|$\\
\phantom{$|a_n| $} = $|s_n - a + a - s_{n-1}|$\\
\phantom{$|a_n| $} $\leq |s_n - a| + |a - s_{n-1}| < \displaystyle\frac{\epsilon}{2} + \displaystyle\frac{\epsilon}{2} = \epsilon$\\
für $n \geq N + 1$\\
$\Rightarrow a_n \to 0$ für $n \to \infty$\\
%
\sS{Satz, die harmonische Reihe}
$\displaystyle\sum\limits_{k = 1}^{\infty} \frac{1}{k}= 1 + \frac{1}{2} + \frac{1}{3} + ...$ divergiert\\
\\
\underline{Beweisidee:}\\
\\
$\phantom{= }1 + \displaystyle\frac{1}{2} + \displaystyle\frac{1}{3} + \displaystyle\frac{1}{4} + \displaystyle\frac{1}{5} + \displaystyle\frac{1}{6} + \displaystyle\frac{1}{7} + \displaystyle\frac{1}{8} + \displaystyle\frac{1}{9} + ...$\\
$\phantom{\geq }1 + \displaystyle\frac{1}{2} + \displaystyle\frac{1}{4} + \displaystyle\frac{1}{4} + \displaystyle\frac{1}{8} + \displaystyle\frac{1}{8} + \displaystyle\frac{1}{8} + \displaystyle\frac{1}{8} + \displaystyle\frac{1}{16} + ...$\\
$\phantom{= }1 + \displaystyle\frac{1}{2} + \displaystyle\frac{2}{4} + \displaystyle\frac{4}{8} + \displaystyle\frac{8}{16} + ...$\\
$\phantom{= }1 + \displaystyle\frac{1}{2} + \displaystyle\frac{1}{2} + \displaystyle\frac{1}{2} + \displaystyle\frac{1}{2} + ... = \infty$\\
\end{document}\newpage
%%Kopfzeile links bzw. innen
\fancyhead[L]{\calligra {\Large Vorlesung Nr. 7}}
%Kopfzeile rechts bzw. außen
\fancyhead[R]{\calligra 29.10.2012}

% set chapters end sections
\setcounter{chapter}3

\section*{Wiederholung Reihen}
Sei $(n_n)$ eine Folge reeler Zahlen.\\
Die Reihe mit den Gliedern $a_n$ ist die Folge $s_n = a_0 + a_1 + ... + a_n)_{n \in \mathbb{N}}$ \\
Bezeichnung: $\sum\limits_{k=1}^{\infty} a_k$\\
Wenn $S_n \to a$ für $n \to \infty$\\
Schreibe: $\sum\limits_{k = 0}^{\infty} a_k = a$\\
\\
\bsp Geometrische Reihe\\
$|x| = 1 \Rightarrow \displaystyle\sum\limits_{k = 0}^{\infty} x^k = \frac{1}{1-x}$ für $x = 0 setzte 0^0 = 1\\ $\\
Harmonische Reihe\\
$\displaystyle\sum\limits_{k = 1}^{\infty} \frac{1}{k}$ Konvergiert nicht.\\
\sS{Satz Rechenregeln für Reihen}
Seien $\sum\limits_{k = 0}^{\infty} a_k = a$ und $\sum\limits_{k = 0}^{\infty} b_k = b$ zwei konvergente Reihen. Dann:
\begin{enumerate}
\item{$\sum\limits_{k = 0}^{\infty} (a_k + b_k) = a + b$}
\item{Für $c \in \mathbb{R}$ ist $\sum\limits_{k = 0}^{\infty} c \cdot a_k = c \cdot a$}
\end{enumerate}
\bew folgt aus 3.9.\\
\bem Produkte von Reihen sind komplizierter.\\
\underline{Korrektur:}
Primzahlen-Vermutung: es gibt $\infty$ viele Primzahlen $p$ so dass $p + 2$ auch Prim ist.\\
Goldbach-Vermutung: Jede gerade natürliche Zahl ist die Summe von zwei Primzahlen.\\
\chapter{Konvergenzsätze}
Erinnerung: $\mathbb{R}$ ist Dedekind-vollständig. Das heißt, jede nicht-leere nach oben beschränkte Teilmenge $M \subset R$ hat eine kleinste obere Schranke $sup(M)$\\
\phantom{XXX}$\to$ Existenz von Grenzwerten\\
\sS{Definition Monotone Folgen}
Eine Folge $(a_n)_{n \geq 0}$ heißt monoton wachsend, wenn $a{n + 1} \geq a_n$ für alle $n \in \mathbb{N}_0$\\
\phantom{Eine Folge $(a_n)_{n \geq 0}$ heißt}monoton fallend, wenn $a{n + 1} \leq a_n$ für alle $n \in \mathbb{N}_0$\\
\phantom{Eine Folge $(a_n)_{n \geq 0}$ heißt}streng monoton wachsend, wenn $a{n + 1} > a_n$ für alle $n \in \mathbb{N}_0$\\
\phantom{Eine Folge $(a_n)_{n \geq 0}$ heißt}streng monoton fallend, wenn $a{n + 1} < a_n$ für alle $n \in \mathbb{N}_0$\\
\\
\bsp
$a_n = n$ ist streng monoton wachsend\\
$a_n = \frac{1}{n}$ ist streng monoton fallend\\
\sS{Satz}
\begin{enumerate}
\item{Jede nach oben beschränkte monoton wachsende Folge $(a_n)_{n \in \mathbb{N}}$ ist konvergent\\
% Bild?
Hier Fehlt was, das Bild, der Tafel, auf dem das stehen sollte ist nicht auffindbar, hast du da noch eine Mitschrift?
}
\item{Jede nach unten beschränkte monoton fallende Folge $(a_n)_{n \in \mathbb{N}}$ ist konvergent\\
% Bild?
}
\end{enumerate}
\bew
Sei $(a_n)$ nach oben beschränkt, monoton wachsend\\
Setze $a:= sup(\{a_n | n \in \mathbb{N}\})$\\
dann \begin{enumerate}
\item{$a_n \leq a$ für alle $n$}
\item{Für jedes $\epsilon > 0$ ist $a - \epsilon$ \underline{keine} obere Schranke, d.h. es gibt $N \in N$ so dass $a_N > a - \epsilon$
\\Für $n \geq N$ gilt\\
$a - \epsilon < a_N \leq a_n \leq a$\\
weil $(a_n)$ monoton wachsend\\
$\Rightarrow a - \epsilon < a_N \leq a_n \leq a \Rightarrow |a_n -a| < \epsilon$\\
Somit $a_n \to a$ für $n \to \infty$\phantom{XXX}$q.e.d.$\\
Monoton fallend: analog}
\end{enumerate}
\underline{\section*{Reihen mit nicht-negativen Gliedern}}\\
\bem Sei $\displaystyle\sum\limits_{k=0}^{\infty} a_k$ Reihe reeller Zahlen\\
Die Folge der Partialsummen ist monoton wachsend $\Leftrightarrow a_n \geq 0$ für $n \geq 1$\\
\sS{Satz}
Eine Reihe $\displaystyle\sum\limits_{k=0}^{\infty} a_k$ mit $a_k \geq 0$ für ale $k$ konvergiert, genau dann, wenn sie beschränkt ist (Das heißt die Folge der Partialsummen ist beschränkt) $q.e.d.$\\
\sS{Definition}
Sei $\displaystyle\sum\limits_{k=0}^{\infty} a_k$ eine Reihe mit $a_n \geq 0$ für alle $k$\\
Eine Reihe $\displaystyle\sum\limits_{k=0}^{\infty} b_k$ heißt \underline{Majorante} von $\displaystyle\sum a_k$ wenn $a_k \leq b_k$ für alle $k$\\
\sS{Satz Majorantenkriterium}
Wenn eine Reihe mit nicht-negativen Gliedern eine konvergente Majorante hat, dann konvergiert sie.\\
\\
\bew
Sei $0 \leq a_k \leq b_k$ für alle $k \geq 0$\\
Es gilt $a_0 + ... + a_n \leq b_0 + ... + b_n$ \\
$\sum b$ konvergiert $\Rightarrow (b_0 + ... + b_n)_{n \geq 0}$  beschränkt\\
$\Rightarrow ((a_0 + ... + a_n)_{n \geq 0})$ beschränkt $\Rightarrow \displaystyle\sum\limits_{k= 0}^{\infty} a_k$ konvergiert.\\
\\
\Bsp{4.6:} $\displaystyle\sum\limits_{k=1}^{\infty} \frac{1}{k^2} = \left( 1 + \frac{1}{4} + \frac{1}{9}+ \frac{1}{16} + ... \right)$\\
$\displaystyle\sum\limits_{k=1}^{\infty} \frac{1}{k^2} = 1 + \displaystyle\sum\limits_{k=1}^{\infty} \frac{1}{(k + 1)^2}$\\
$\frac{1}{(k + 1)^2} \leq \frac{1}{k \cdot (k + 1)}$\\
$\Rightarrow \displaystyle\sum\limits_{k=1}^{\infty} \frac{1}{k \cdot (k + 1)}$ ist Majorante von $\displaystyle\sum\limits_{k=1}^{\infty} \frac{1}{k^2}$\\
$\displaystyle\sum\limits_{k=1}^{\infty} \frac{1}{k \cdot (k + 1)}$ konvergiert (Bekannt)\\
\sS{Satz Quotientenkriterium}
Sei $C \in \mathbb{R}, (a_n)$ eine Folge reeller Zahlen mit $a_n \geq 0$ für alle $n$ \underline{und} $a_{n + 1} \leq C \cdot a_n$ für fast alle $n$\\
$0 \leq C \leq 1$\\
Dann konvergiert die Reihe $\displaystyle\sum\limits_{k=0}^{\infty} a_k$\\
\\
\bew
Konvergenz ändert sich nicht, wenn endlich viele $a_n$ geändert werden.\\
Also kann man annehmen, dass $a_{n + 1} \leq C \cdot a_n$ für alle $n$ gilt.\\
\\
Dann gilt $a_1 < C \cdot a_0$\\
$a_2 < C \cdot a_1 \leq C \cdot C \cdot a_0 = C^2 \cdot a_0$\\
$a_3 < C \cdot a_2 \leq C \cdot C \cdot a_1 = C^3 \cdot a_0$\\
etc. $\Rightarrow a_n \leq C^n \cdot a_0$\\
Somit ist $\displaystyle\sum\limits_{k=0}^{\infty} C^k \cdot a_0$ konvergente Majorante von $\displaystyle\sum\limits_{k=0}^{\infty} a_k$ (Geometrische Reihe)\\
\sS{Beispiel Die Exponentialreihe}
$exp(x) := \displaystyle\sum\limits_{k=0}^{\infty} \frac{x^k}{k!}$ für $x \in \mathbb{R}, x \geq 0$\\
\\
Setze $a_k = \frac{x^k}{k!}$\\
$a_n+1 = \frac{x^{n + 1}}{(n + 1)!} = \frac{x}{n+1} \cdot \frac{x^n}{n!} = \frac{x}{n+1} \cdot a_n \leq \frac{1}{2} a_n$\\
$\Rightarrow$ Quotientenregel ist erfüllt.\\
\\
Reihe exp(x) konvergiert.\\
\\
Bezeichnung: $exp(x) = \displaystyle\sum\limits_{k=0}^{\infty} \frac{x^k}{k!} \in \mathbb{R}$\\
\\
\underline{\subsection*{Bezeichnung: }} $exp(1) = \displaystyle\sum\limits_{k=0}^{\infty} \frac{1}{k!} = e$ (Eulerische Zahl)
\sS{Leibnitz-Kriterium}
Sei $(a_n)_{n \in \mathbb{N}_0}$ eine Monoton monoton fallende Nullfolge\footnote{$a_n \to 0 \text{ für } n \to \infty$} mit $a_n \geq 0$ für alle $n$\\
Dann konvergiert die alternierende Reihe\\
$\displaystyle\sum\limits_{k=0}^{\infty} (-1)^k \cdot a_k$\\
\bsp\\
$a_k = \frac{1}{k + 1} \displaystyle\sum\limits_{k=0}^{\infty} (-1)^k \cdot a_k = 1 - \frac{1}{2} + \frac{1}{3} - \frac{1}{4} + \frac{1}{5} \\
= log(2)$
\sS{Beweis}
Sei $s_n = a_0 + ... + a_n$\\
\underline{\subsection{Behauptung: }}\\
$S_{2n + 1} \leq S_{2n + 3} \leq S_{2n + 2} \leq S_{2n}$ für jedes $n \in \mathbb{N}$\\
\\
 
\noindent\underline{Rechne:}\\
$S_{2n + 2} - S_{2n} = - a_{2n + 1} + a_{2n + 2} \leq 0 \Rightarrow (3)$\\
$S_{2n + 3} - S_{2n + 1} = - a_{2n + 3} \leq 0 \Rightarrow (2)$\\
$S_{2n + 3} - S_{2n + 1} = - a_{2n + 2} - a_{2n + 3} \leq 0 \Rightarrow (1)$\\
\\
Die Folge $b_n = S_{2n}$\\
\phantom{Die Folge }$c_n = S_{2n + 1}$\\
sind beschränkt und monoton (fallend bzw. steigend)\\
$\Rightarrow b_n \text{ und } c_n$ konvergieren\\
\\
Sei $b = \displaystyle\lim_{n \to \infty} b_n \phantom{XXX} c = \displaystyle\lim_{n \to \infty} c_n$\\
$c - b = \displaystyle\lim_{n \to \infty} (c_n - b_n) = \displaystyle\lim_{n \to \infty} (a_{2n + 1}) = 0$\\
weil $(a_n)$ Nullfolge\\
\\
\underline{\subsection*{Behauptung:}} $S_n \to b$ für $n \to \infty$\\
\\
Gegeben sei $\epsilon > 0$. Es gibt $N \in \mathbb{N}$ so dass für $n \leq N$:\\
$|b_n - b| < \epsilon, |c_n - c| < \epsilon$\\
Somit für $n \geq 2N+1 $\\
$|S_n - b| < \epsilon$ also $S_n \to b \phantom{XXX} q.e.d.$
\end{document}

\newpage
%\input{v8}\newpage
%\input{v9}\newpage
%% Kopfzeile beim Kapitelanfang:
\fancypagestyle{plain}{
%Kopfzeile links bzw. innen
\fancyhead[L]{\calligra\Large Vorlesung Nr. 10}
%Kopfzeile rechts bzw. außen
\fancyhead[R]{\calligra\Large 12.11.2012}
}
%Kopfzeile links bzw. innen
\fancyhead[L]{\calligra {\Large Vorlesung Nr. 10}}
%Kopfzeile rechts bzw. außen
\fancyhead[R]{\calligra \Large{12.11.2012}}
% **************************************************
%
%\setcounter{chapter}{4}
\chapter{Abbildungen und Funktionen}
\sS{Definition}
Seien $A, B$ Mengen. Eine Abbildung von $A$ nach $B$ ist eine Vorschrift, die jedem Element von $A$ ein Element von $B$ zuordnet.\\
\notat{$f: A \to B,\  a \mapsto f(a) \  a\in A$}
%
A heißt Definitionsbereich von $f$\\
B heißt Wertebereich von $f$
%
\bsp
\begin{enumerate}
\item {Alle Personen in $L1 \mapsto \N$\\
$P \mapsto$ Geburtsjahr von $P$}
%
\item{$f:\R → \R, \ f(x)=x^2$\\
$g:\R→\R_{\geq 0}=\{x\in\R|x\geq 0\}, \ g(x)=x^2$\\
$h: \R_{\geq 0} \to \R_{\geq 0} \ h(x) = x^2$}
\bem 
\item{
$f,g,h$ sind verschieden\\
Sei $M$ Menge. Die Identität von $M$ ist die Abbildung $id_{M}:M→M, \ id_M(x)=x$}
\end{enumerate}
%
\sS{Definition}
%
Eine Abbildung $f: A \to B$ heißt:
%
\begin{enumerate}
\item{\underline{injektiv} wenn gilt: Für alle $a, a' \in A$ mit $f(a) = f(a')$ ist auch $a = a'$}
\item{\underline{surjektiv} wenn gilt: Für jede $b\in B$ gibt es ein $a\in A$ mit $f(a)=b$}
\item{\underline{bijektiv} wenn $f$ injektiv und surjektiv ist}
\end{enumerate}
%
% Tafel 2.2
% Bild zeichnen
%
% Tafel 3.1
% Beispielbild
%
\bem
$f$ ist $\left\{
\begin{array}{c}
\text{injektiv}\\
\text{surjektiv}\\
\text{bijektiv}
\end{array}
 \right\}$ genau dann wenn für jedes $b \in B$ $\left\{\begin{array}{c} \text{höchstens}\\ \text{mindestens}\\ \text{genau} \end{array} \right\}$ ein $a \in A$ mit $f(a) = b$\\
%
\bsp
$f,g,h$ wie oben\\

\begin{description}
\item[f]{
\hspace{6mm}nicht surjektiv: es gibt kein $a\in\R$ mit $f(a)=-1$\\
nicht injektiv: $f(-2)=4=f(2), 2\neq -2.$}
%
\item[g]{
\hspace{5mm}ist surjektiv, denn für jedes $b \in \R_{\geq 0}$ gilt $f( \sqrt{b} ) = b$ also gibt es $b \in \R_a$\\
 ist nicht injektiv (wie $f$)}
%
\item[h]{
\hspace{5mm}surjektiv wie g. $(\sqrt{b} \geq 0)$\\
injektiv, denn: Wenn $a, a' \geq 0$ und $a^2 = (a')^2$ dann $a = a'$ also $h$ bijektiv.}
\end{description}
%
\sS{Definition}
Seien $f:A→B$, $g:B→C$ Abbildungen\\
Die Komposition von $f$ und $g$ ist die Abbildung\\
$g \circ  f: A→C$, $(g \circ f)(a):=g(f(a))$\\
Sprich $\circ$ "nach"
%
%
\sS{Satz} 
Eine Abbildung $f: A \to B$ ist bijektiv \equ \ es gibt eine Abbildung $g: B \to A$ mit $f \circ g = id_B$\\
(d.h. f(g(b)) = b für alle $b \in B$\\
      g(f(a)) = a für alle $a \in A$)\\
%
\sss{Definition} %ohne nummer
Wenn $f:A→B$ bijektiv ist, heißt die eindeutige Abbildung $g:B→A$ wie oben die Umkehrabbildung (inverse Abbildung) von $f$
Bezeichnung: $g=f^{-1}$.\\
%
\bew
Angenommen, $g: B \to A$ gegeben mit $f \circ g = id_B, g \circ f = id_A$ \footnote{Dies gilt, weil $g$ als Umkehrfunktion von $f$ definiert ist.}\\
$f$ surjektiv: Sei $b \in B$. $b = f(g(b)) = f(a)$ mit $a = g(b)$ \ok\\
$f$ injektiv: Sei $a, a'$ mit $f(a) = f(a')$ zeige $a = a'$ \\
$a = g(f(a)) = g(f(a')) = a' $\ok \\ \\
%
Angenommen, $f$ ist bijektiv, zeige $g$ existiert.\\
Gegeben sei $b \in B$ $f$ bijektiv $\Rightarrow$ es gibt genau ein $a \in A $ mit $f(a) = b$ 
Setze $g(b):=a$ Das definiert Abbildung $g:B→A$\\
Zeige $g \circ f=id; f \circ g = id$\\
$(f\circ g)(b)=f(g(b))=f(a)=b$ wobei $a$ wie eben\\ \\
%
Zeige: $(g \circ f) (a) $ für alle $a \in A$\\
$f$ injektiv: Reicht $f(g(f)a))) = f(a)$\\
Das gilt weil $f \circ g = id_B$ \ok \\ \\
Eindeutigkeit von $g$:\\
Angenommen, $g^* : B \to A$ erfüllt $g^* \circ f = id_A$,
$f \circ g^* = id_B$ \\
%
Dann gilt: $g=g\circ id_B=g\circ f\circ g^*=id_A\circ g^* = g^*$ \qed
\bsp
Bewiesen 5.12
\begin{itemize}
\item{$f: \R_{\geq 0} \to \R_{\geq 0}, f(x) = x^k$ bijektiv ($k \geq 1$)\\
Die Umkehrabbildung $f^{-1}$ heißt k-te Wurzelabbildung $f^{-1}(x) = \sqrt[k]{x}$ }
%
\item{exp: $\R→\R_{>0}$ $exp(x) = \sum_{k=0}^{\infty}$ (Absolutkonvergente Reihe) ist bijektiv. Die Umkehrabbildung heißt Logarithmus. bew.
$log = exp()^{-1} \R_{\geq } \to \R_a$ }
\end{itemize}
%
\section*{Bild und Urbild}
\sS{Definition}
Sei $f:A→B$ Abbildung\\
\begin{enumerate}
\item{Für eine Teilmenge $X \subset A$ ist \\
$f(x) := \{f(x) | x \in X\} \subseteq B$ \\
das Bild von $X$ unter $f$}
\item{Für eine Teilmenge $Y \subseteq B$ ist $f^{-1}:=\{a\in A|f(a)\in Y\}\subseteq A$ das Urbild von $Y$ unter $f$}
\end{enumerate}
\underline{\underline{Vorsicht}} nicht Urbild und Umkehrabbildung verwechseln.\\
\bsp
$f:\R→\R, f(x)=x^2$\\
$f(\{1, 2, -2\}) = \{1, 4\}$\\
$f^{-1}(\{1,-2,4\})=\{1,-1,2,-2\}=f^{-1}(\{1,4\})$\\
$f^{-1}(\{9\})=\{3,-3\} \\
f^{-1}(\{-5\})=\emptyset$
%
\section*{Funktionen}
\sS{Definition}
Sei $D\subseteq\R$ Teilmenge. Eine reelle Funktion auf $D$ ist eine Abbildung $f:D→\R$\\
%
Der \underline{Graf} von $f$ ist die Menge $\Gamma_f = \/(x, f(x) | x \in D \}$) \\
$ \Gamma_f \subseteq D \times \R$ 
%
\bem Oft ist $D$ ein Intervall
%
\sS{Definition Intervalle}
\begin{wrapfigure}{r}{0.2\textwidth}
  \begin{center}
\begin{tikzpicture}[scale=.5, domain=-3:3, samples=2000]
    \draw[very thin,color=gray] (-3.5,-0.0) grid (3.5,9.0);
    \draw[->] (-3.5,0) -- (3.5,0) node[right] {$x$};
    \draw[->] (0,-0.2) -- (0,9.2) node[above] {$f(x)$};
    \draw[color=red] plot[id=abb1.1] function{x*x} 
        node[right] {$f_1(x) =x^2$};
    \draw[color=blue] plot[id=abb1.2] function{abs(x)} 
        node[right] {$f_2(x) = |x|$};
    \draw[color=cyan] plot[id=abb1.4] function{exp(x)} 
        node[right] {$f_4(x) = exp(x)$};
    \draw[color=black] plot[id=abb1.5] function{floor(x)} 
        node[right] {$f_5(x) = [x]$};
\end{tikzpicture}
  \end{center}
\end{wrapfigure}
seien $a, b \in \R$ \\
$[a, b] = \{x \in \R| a \leq x \leq b\}$ (abgeschlossen)\\
$(a, b] = \{x \in \R| a < x \leq b\}$ (halboffen)\\
$[a, b) = \{x \in \R| a \leq x < b\}$ (halboffen)\\ %mit klammer zu pberer zeile\\
$(a, b) = \{x \in \R| a < x < b\}$ (offen)\\
%
Uneigentliche Intervalle: \\
$[a, \infty) = \{x \in \R | a \leq x\} = \R_{\geq a}$\\
$(a, \infty) = \{x \in \R | a < x\} = \R_{> a}$\\
$(- \infty, a] = \{x \in \R | x \leq a\} = \R_{\leq a}$\\
$(- \infty, a) = \{x \in \R | x < a\} = \R_{< a}$\\
$(- \infty, \infty) = \R$\\
%
\Bsp{Funktionen}
\begin{enumerate}
\item{$f:[0,2]→\R, f(x)=x^2, \Gamma_f \leq [0,2] x\R$}
\item{Betragsfunktionen: $|\ |: \R→\R, x\mapsto|x|$
%noch mehr graphen aaaahahhahahah
}
An dieser Stelle fehlen noch Graphen.
\item{$g:\R\bs\{0\}→\R, g(x)=\dfrac{1}{x}$
%graph
Hier auch.
}
\item{$exp:\R→\R$.}
\item{[.] : $\R \to \R$ Gaußklammer\\
$[x] := max\{n \in \Z | n \leq x \}$
\bsp
$[5] = 5$\\
$[5,78] = 5$\\
$[-1,2] = -2$}
\item{Sei $h:\R→\R$ definiert durch $h(x)=\begin{cases}0\ wenn\ x\in\Q\\ 1\ wenn\ x\notin\Q\end{cases}$\\
$h(\sqrt{2}) = 1, h (\frac{3}{7}) = 0$}
\end{enumerate}
%
\sS{Definition (Rechnen mit Funktionen)}
Sei $D \subseteq \R , \ f,g: D→\R$ Funktionen auf D.\\
Definiere
\begin{itemize}
\item{$f+g: D \to \R$ durch $(f + g)(x) := f(x) + g(x)$}
\item{$(f \cdot  g) (x) = f(x) \cdot g(x)$}
\item{Für $a\in\R$ setze $a·f: D→\R, (a·f)(x):=a·f(x)$}
\item{Angenommen, $f(x) \neq 0$ für alle $x \in D$ \\
$$\frac{1}{f}: D \to R, \frac{1}{f}(x) := \frac{1}{f(x)} = f(x)^{-1}$$
\underline{\underline{Vorsicht}} nicht $\frac{1}{f}$ mit Umkehrbild oder Urbild verwechseln}
\end{itemize}
%
\sS{Definition}
\begin{itemize}
\item{Eine \underline{Polinomfunktion} ist eine Funktion der Form\\
$f: \R → \R,\ f(x) = a_n x^n+a_{n-1}x^{n-1}+…+a_0=\ds\sum_{k=0}^n a_k x^k $\\
wobei $a_0,…,a_n \in \R$ fest}s
%
\item{Seien $f, g : \R \to \R $ Polymonfunktionen
Sei $D = {x \in \R | g(x) \geq 0}\leadsto \dfrac{f}{g} : D \to , Rx \mapsto \frac{f(x)}{g(x)}$
Solche Funktionen heißen rationale Funktionen.
\bsp
$f:\R\bs\{0,1\}→\R, \ f(x)=\dfrac{x^7+5x^2}{x(x-1)}$}
\end{itemize}
\sS{Definition}
Seien $f: C \to \R, g: D \to \R$ Funktionen sodass $f(C) \subseteq D$
Eine Komposition von f und g ist 
%
$g \circ f : C \to \R$\\
$(g \circ f) \ (x) = g(f(x))$\newpage
\wdh
Eine Abbildung $f:x→y$\\
\begin{itemize}
\item{ist \underline{injektiv} wenn gilt:\\
für alle $a,b\in X$ mit $f(a)=f(b)$ ist $a=b$}
\item{ist \underline{surjektiv} wenn für jedes $y\in Y$ ein $a\in X$ existiert mit $f(a)=y$}
\end{itemize}}
Sei $D\subseteq\R$ Teilmenge. Eine Funktion auf D ist eine Abbildung $f:D→\R$
%
\uS{Monotone Funktionen}
\bem
Eine Funktion $(a_n)_{n\geq 0}$ reeler Zahlen ist eine Abbildung $a:\N_0→\Re$ d.h. eine Funktion auf $\No_0$
%
\def
Sei $D\subseteq\R$. Eine Funktion  $f:D→\R$ heißt:
\begin{enumerate}
\item{\underline{monoton wachsend} wenn gilt:\\
Für alle $a,b\in D$ mit $a<b$ ist immer $f(a)\leq f(b)$}
\item{\underline{streng monoton wachsend}: $a<b\Rarr f(a)<f(b)$}
\item{\underline{monoton fallend}: $a<b\Rarr f(a)\geq f(b)$}
\item{\underline{streng monoton fallend}: $a<b\Rarr f(a)> f(b)$}
\end{enumerate}
%
\bem
Jede streng monotone Funktion f ist injektiv
%
\bew
Zeige: $a\neq b\Rarr f(a)\neq f(b)$\\
Wenn $a\neq b$ dann $a< b$ oder $b<a$\\
Wenn f streng monoton wachsend: Folgt $f(a)< f(b)$ oder $f(b)< f(a)$ also $f(a)\neq f(b)$\\
Wenn f streng monoton fällt: es folgt $f(a)> f(b)$ oder $f(b)> f(a)$ also $f(a)\neq f(b)$\qed
%
%\sS{Beispiel}
%\begin{enumerate}
%\item{$f:\R_{\geq 0}→\R,\ x\mapsto x^k =:f(x)$ mit $k\geq 1$\\%umgekehrtes define
%f ist streng monoton wachsend/steigend%bild tafel 2.2
%\end{enumerate}
%
% Cristopher: Tafel 2.2.2 bis 5.2.1
%
\chapter{Stetigkeit}
\underline{Idee:} Eine Funktion ohne sprünge heißt \underline{stetig}\\
\def
Sei $D\subseteq \R,\ f:D→\R$ eine Funktion\\
\begin{enumerate}
\item{f heißt stetig in $x\in D$ wenn gilt:\\
Für jedes $\e>0$ gibt es ein $\delta>0$ so dass für jedes $y\in D$ mit $|x-y|<\delta$ gilt $|f(x)-f(y)<\e$ %graph 6.2}
\item{f heißt stetig wenn f in jedem $x\in D$ stetig ist}
\end{enumerate}
\sS{Beispiel}
\begin{enumerate}
\item{Die Funktion $id:\R→\R,\ x\mapsto x$ ist stetig}
\item{Die Funktion $f:\R→\R,\ f(x)=x^2$ ist stetig. %graph klein
\bew
Sei $x,y\in\R\quad y=x+k$.\\
$$f(y)-f(x)=(x+h)^2-x^2=x^2+2xh+h^2-x^2=2xh+h^2$$\\
Wähle jedenfalls $\delta\leq 1$. Wenn $|h|=|x-y|>\delta$ dann $|h|<1$\\
$$|f(y)-f(x)|=|2xh+h^2|\leq|2x|·|h|+|h|^2<|2x|·|h|+|h|=(|2x|+1)·|h|$$\\
Gegeben sei $\e>0$\\
Wähle $\delta=min\left\{1,\dfrac{\e}{|2x|+1}\right\}$\\
Wenn $|x-y|<\delta$ dann $$|f(x)-f(y)|<(2|x|+1)·|h|<(2|x|+1)·\dfrac{\e}{2|x|+1}=\e$$\\
Also $f$ stetig in $x$}
\item{$g≔\R→\R,\ g(x)≔\{x\}$\\ %graph
g ist stetig an $x$ \equ $x\notin\Z$
\Bew{g nicht stetig an $x\in\Z$:}
Zeige: es gibt ein $\e>0$ so dass kein $\delta>0$ existiert mit: $|x-y|>\delta\Rarr|g(x)-g(y)|<\e$\\
z.B. $\e=1$ Sei $\delta>0.\ y=x-\dfrac{\delta}{2}\quad |x-y|=\dfrac{\delta}{2}<\delta$\\
aber $g(y)=\{x-\dfrac{\delta}{2}\}=x-1$ (weil $x\in\Z$)\\
|g(x)-g(y)|=|x-(x-1)|=1 \not<\e\qed}
\end{enumerate}
%
% Christopher
%
\sS{Satz (Folgenstetigkeit)}
Sei $D\subseteq\R,\ x\in D,\ f:D→\R$ Funktion $f$ ist genau dann stetig in $x$ wenn gilt:\\
\begin{itemize}
\item{Für jede Folge $(x_n)_{n\geq 0}$ mit $x_n\in D,\ x_n→x$ für $n→∞$ gilt auch $f(x_n)→f(x)$ für $n→∞$}
\end{itemize}
%
\satz
Sei $D\subseteq\R,\ f,g:D→\R$ in $x\in D$\\
Dann gilt:\\
\begin{itemize}
\item{$f+g:D→\R$ stetig in $x$}
\item{$f·g:D→\R$ stetig in $x$}
\item{Wenn $g(x)\neq 0$ für alle $x'\in D$}
\end{itemize}
Dann ist $\dfrac{1}{f}:D→\R$ stetig in x.
\bew
mit Folgenstetigkeit
%letzte seite\newpage
\end{document}


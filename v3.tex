%Kopfzeile links bzw. innen
\fancyhead[L]{\calligra {\Large Vorlesung Nr. 1}}
%Kopfzeile rechts bzw. außen
\fancyhead[R]{\calligra \Large{8.10.2012}}
% **************************************************
%
\wdh
Ein Körper K ist eine Menge mit $+$ und $\cdot$, sodass gewisse Eigenschaften erfüllt sind:
\bsp
$\ds\Q = \left\{\frac{a}{b} | a \in \Z, b \neq 0\right\}$\\
$F_1 = \{0, 1\} \qquad 1 + 1 = 0$\\
\notat{Setze $a^n = \underbrace{ a · a · a · a· … · a}_{n-Faktoren}$\\
$\left.\begin{array}{lcc}
a^0 &=& 1\\
a^{-n} &=& (a^{-1})^n
\end{array}\right\}$ wenn $a \neq 0$}
Daraus folgt $a^n$ ist definiert, wenn $a \neq 0$ und $n \in \Z$\\
Regeln der Potenzgleichung:\\
$a^{n+m} = a^n \cdot a^m$\\
$a^{n \cdot m} = (a^{n})^m$\\
\bew
Übung
%
\sS{Definition}
    Ein angeordneter Körper ist ein Körper K für dessen Elemente eine "Kleiner als Beziehung" $<$ definiert ist, so dass folgende Eigenschaften erfüllt sind:\\
    \begin{enumerate}
    \item{Für alle $a, b \in K$ gilt genau eine von drei Notationen:\\
    $a < b oder a = b oder a > b$}
    \item{Für alle $a, b, c \in K$ gilt wenn $a < b$ und $b < c$ dann $a < c$\\ (Transitivität)}
    \item{Für alle $a, b, c \in K$ gilt wenn $a < b$ dann $a + c < b + c$}
    \item{für $a, b, c \in K$ gilt, wenn $a < b$ und $c \neq 0$ dann $a \cdot c < b \cdot c$}
    \end{enumerate}
	Weitere Beziehungen:\\
	$a > b$ heißt $b < a$\\
	\begin{enumerate}
	\item{Wenn $a < 0$ dann $-a > 0$:\\
	$a < 0 \Rarr a + (-a) > 0 + (-a) \Rarr 0 > -a$}
	\item{Für jedes $a \in K $ gilt wenn $a \neq 0$, dann $a^2 > 0$\\
	\begin{itemize}
	\item[(a)]{$\begin{array}{ccc}
	a &>& 0\\
	a · a &>& 0 · a\\
	a^2 &>& 0
	\end{array}$\qed}
	\item[(b)]{$\begin{array}{ccc}
		a &<& 0\\
		-a &>& 0 · a\\
		a^2 &=& (-a)^2 > 0
		\end{array}$\qed}
	\end{itemize}}
	\item{$1 > 0 $ denn $1 = 1^2$}
	\end{enumerate}
	Sei $K$ ein Angeordneter Körper:\\
	$0 < 1 \Rarr 1 < 1 + 1 \Rarr 1 + 1 < 1 + 1 + 1 $ etc.\\
	$0 < 1 < 1 + 1 < 1 + 1 +1 $ etc.
	Für \nN{} setze $underbrace{n:= 1 + 1 + 1 + … + 1}$
	